\chapter*{概要}
  太陽活動に起因する激しい宇宙天気の変動は、地球上の電力網や衛星通信などの技術システムに深刻な影響を与える可能性がある。
  そのため、太陽活動の観測と宇宙天気の予測は、人類の社会活動を守る上で重要であり、その予測モデルの開発や、現象のメカニズムの研究は盛んに行われている。
  宇宙天気予報において、NASAのSolar Dynamics Observatory (SDO) をはじめとする観測衛星から得られる観測データが重要な役割を果たしている。
  これらのデータは予測モデルの入力や、専門家による予報の情報源として利用されている。
  
  近年の深層学習の進展により、宇宙天気予報においてもその応用が見られるようになった。
  深層学習モデルは迅速な予測を可能にし、確率論的なアプローチを通じて不確実性を考慮することができる。
  NICTによるDeep Flare Netや、NASAによるDeep Learning Geomagnetic perturbationなど、深層学習を利用した予測モデルは実際に運用されており、宇宙天気予報においてその有用性が確認されている。
  
  宇宙天気による影響から重要な技術システムを保護するには、早期かつ正確な予測が必要である。
  しかし、シミュレーションによる予測モデルや深層学習による予測モデルは、多くの場合、数時間から数日後までの予測に焦点を当てている。
  さらに長い時間範囲での予測が理想であるが、計算資源の制約や、太陽の複雑なシステムに対するモデルの予測能力の限界などの理由により、その実現は困難である。

  そのような背景から、本研究では、深層学習を用いた、未来の太陽画像の生成という新しいアプローチを提案する。
  具体的には、動画予測という技術を用いて、SDO / AIAから得られる全球紫外線画像の予測を試みる。
  本研究により、未知の太陽画像を精度よく再現することができれば、既存の予測モデルの入力データとして利用することができ、より早期の宇宙天気予報の実現に貢献することができると考えられる。

  動画予測とは、動画の一部を入力として、それに続く未来の動画を予測するタスクである。
  CNNを中心とした画像処理技術の進展と、LSTMを中心とした時系列データの処理技術の進展、およびその融合により、近年注目を集めている。
  
  はじめに、Motion-Aware Unitと呼ばれる動画予測モデルを用いて、SDO / AIAの211\AA フィルターから得られた時系列全球データを入力とし、48時間以内の4時間ごとの全球紫外線画像を生成するモデルを構築した。
  生成された予測画像に対し、全球、経度ごと、さらに東側外縁部における輝度強度の再現性を評価した。
  さらに、より詳細な性能評価と比較のために、単純作動回転モデルとの比較を行った。
  全球での平均輝度強度の絶対誤差は、48時間後の予測で3.67\% であり、単純差動回転モデルの絶対誤差である10.2\% よりも低かった。
  経度ごとの評価でも、すべての分割セクターにおいて、48時間後時点での絶対誤差が単純差動回転モデルよりも低かった。
  また、東側外縁部においても、良好な性能を示し、動画予測モデルの高い学習能力が確認された。
  さらに、動画予測モデルが予測に要求する時間は数秒であり、非常に高速に予測画像を生成することができた。

  次に、さらなる性能向上を目的として、211 \AA フィルターの全球画像に加え、171 \AA フィルター、193 \AA フィルターの全球画像を追加し、3波長の入力から211 \AA フィルターの画像を予測するモデルを構築した。
  評価は全実験と同様に行った。
  この実験では、ほとんどの評価指標において、実験1の結果を下回るか、同等の結果となった。
  これは、現在使用してる動画予測モデルのアーキテクチャやモデルの深さでは、3波長の入力に対して十分な学習を行うことができないことが原因と考えられる。

  このような結果から、深層学習を用いた動画予測手法は、未知の太陽全球紫外線像の予測という課題において、有効な手法であることが示された。
  特に、高い時空間的ロバスト性を実現する学習能力や、予測の高速性は、実際の宇宙天気予報において求められる重要な条件である。
  今後、本研究の結果をもとにしたさらなるアプローチや改善手法の検討を行うことで、本研究の成果をさらに発展させることができると考えられる。

