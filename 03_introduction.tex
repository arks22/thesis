\chapter{研究背景}

 宇宙天気の地球に対する影響
 宇宙天気の撹乱の原因となる太陽活動
 太陽活動の観測と宇宙天気予報
 宇宙天気予報における紫外線画像の利用と重要性
 紫外線像を利用した宇宙天気予報モデルの先行研究
 研究目的: 数日後の紫外線画像を予測し生成する
 近年の動画予測モデルの登場と発展
 本研究で用いる動画予測モデルの概要

宇宙天気とは、太陽活動に起因する宇宙環境の現象を指し、激しい宇宙天気の変動は地球上の電力網や衛星通信など、人間の技術システムに影響を及ぼすことがある。
太陽の表面や大気における物理現象、特に太陽フレアやコロナ質量放出(CME)などの爆発的なイベントは、地球に到達する高エネルギー粒子や放射線の量を増加させ、宇宙天気の変動に大きな影響を与えることが知られている。
そのため、太陽活動を観測し、宇宙天気を予測することは、人間の技術システムを宇宙天気の影響から守るための重要な課題である。

宇宙天気の予報には、太陽フレアの爆発やCMEの発生を予測するものや、太陽風の地球到達時刻や強度を予測するものなどがある。
これらの予測は、様々な観測機器によって得られる太陽観測データを用いて行われるが、紫外線像はその中でも重要な情報源の一つである。
(DeFN、WindNetの話とか)