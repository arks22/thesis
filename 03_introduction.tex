\chapter{研究背景}

宇宙天気とは、太陽活動に起因する宇宙環境の現象を指し、激しい宇宙天気の変動は地球上の電力網や衛星通信など、人間の技術システムに影響を及ぼすことがある。
太陽の表面や大気における物理現象、特に太陽フレアやコロナ質量放出(CME)などの爆発的なイベントは、地球に到達する高エネルギー粒子や放射線の量を増加させ、宇宙天気の変動に大きな影響を与えることが知られている。
そのため、太陽活動を観測し、宇宙天気を予測することは、人間の技術システムを宇宙天気の影響から守るための重要な課題である。

宇宙天気の予報には、太陽フレアの爆発やCMEの発生を予測するものや、太陽風の地球到達時刻や強度を予測するものなどがある。
これらの予測は、様々な観測機器によって得られる太陽観測データを用いて行われるが、紫外線像はその中でも重要な情報源の一つである。
(DeFN、WindNetの話とか)
このように、太陽の紫外線像、およびそれから得られる特徴量は、宇宙天気予報において重要な役割を果たしている。
そこで、まだ観測されていない未来の紫外線画像を予測、生成することで、より早期の宇宙天気予報の実現に貢献できるのではないかと考えた。

近年、深層学習技術の発展により、「動画予測(Video Prediction)」と呼ばれる技術が注目されている。動画予測とは、動画の一部を入力として、それに続く未来の動画を予測するタスクである。
動画予測を行う深層学習モデルのアーキテクチャの多くは、Long Short-Term Memory(LSTM)などの再帰的ニューラルネットワーク(RNN)のアーキテクチャを基本とする。さらに、特徴量抽出および伝播にConvolutional Neural Network(CNN)を用いることで空間的な特徴を時系列にわたってとらえ、Decoderモデルによって動画を生成する。
このような動画予測モデルは、Conv-LSTMの登場で初めて提案されて以降、様々なモデルが提案されており、自動運転や天気予報など、様々な分野での応用が期待されている。

本研究では、動画予測モデルを用いて、数日後の紫外線画像を予測、生成することを目的とする。
Deep Flare Netなどの多くの深層学習を用いた宇宙天気予報モデル、またそれらを用いた人間による主観的な宇宙天気予報は、現在の観測情報を用いて、数時間後から数日後までの宇宙天気を予測するものが多い。
それらの情報源として、数日後の高精度な紫外線画像を生成することができれば、より早期の宇宙天気予報の実現に貢献できるのではないかと考えた。

そのような数日後の紫外線像の生成のために、本研究ではMotion-Aware Unit(MAU)と呼ばれる動画予測モデルを用いる。
MAUは、RNNやCNNを用いた基本的な動画予測モデルを基本としつつ、各時間時点における画像の処理にMAU Cellと呼ばれるモジュールを多層的に積み重ねたアーキテクチャを採用している。
MAU CellはAttentionと呼ばれる機構を持ち、長期的な依存関係を適切に学習する能力を持つ。また、従来の動画予測モデルで提案されてきたEncoder-Decoderモデルや、メモリフローの改善など、様々な改良が加えられており、多くの動画予測モデルの中でも要求する計算量に対して高い予測精度を達成している。

本研究では、MAUを用いて、数日後の紫外線画像を予測、生成することを目的とするが、その評価には主に輝度強度の再現性を用いる。
これは、宇宙天気予報モデルの多くではその特徴量として輝度強度を用いていること由来する。
