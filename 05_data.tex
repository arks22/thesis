\chapter{データ} 

\section{SDO / AIA}
モデルの学習及び評価データとして、NASAのSolar Dynamic Observatory(SDO)\cite{pesnell2012solar}のAtmospheric Imaging Assembly(AIA)で撮影された紫外線観測データを用いた。

SDOは2010年2月に打ち上げられた太陽観測衛星であり、AIA、Helioseismic and Magnetic Imager(HMI)、Extreme Ultraviolet Variability Experiment(EVE)などの高い空間解像度、時間分解能を持つ観測機器を搭載している。その観測データは太陽物理学、宇宙天気予報、また地球環境に関する研究など、多くの分野で利用されている。\\
なかでも、AIAは主に太陽大気を紫外線で観測する観測機器である。4096×4096、約0.6arcsecの空間解像度をもち、数秒ごとに太陽全球の画像を観測している。また、複数の紫外線波長フィルターで同時に観測を行うことで、多層的な太陽大気の理解に貢献している。\\
これらのデータはJoint Science Operations Center(JSOC) によって提供されており、Pythonの太陽物理学を支援するライブラリであるSunpyを用いてダウンロードすることができる。

実験1では入力、出力ともに211Åフィルターで得られたデータを利用した。これは 211Åフィルターで撮影された紫外線像が、コロナホールと活動領域といった、二つの太陽円盤上の大規模構造をバランスよく明瞭に表現し、本研究のモデルの効果検証に適していると考えたためである。
また、実験2では入力に171Å、193Åフィルターで得られたデータを追加で利用した。これらの波長を追加することで、より広範な温度帯に渡る太陽活動をモデルが学習することを期待している。

\subsection{AIA 171Å}
\subsection{AIA 193Å}
\subsection{AIA 211Å}


\section{データセットの作成}

本研究で用いるデータセットには、SDO/AIA望遠鏡のデータが提供されている2010年5月から、2022年10月までのデータが含まれている。\\
この期間に存在するデータから、4時間ごとにデータを抽出し、約22000枚(正確な方がいい?)をデータセットに含んでいる。これらのデータを、24枚の画像を1セットとして分割する。各セットは24枚の時系列に並んだ画像で構成され、太陽の時間的依存性および空間的情報を同時に捉えている。24枚のうち、前半の12枚、すなわち44時間後までを入力データ、後半の12枚、すなわち48時間後から92時間後までを出力データとして扱う。学習の際は、前半の12枚に対して後半の12枚を教師データとして扱い、テストの際は12枚の画像データに続くモデルにとって未知の12枚を再現できるか検証する。

このようにして作成されたデータセットは、926セットになり、これを学習用データセット=826、検証データセット=50,テストデータセット=50というように分割した。

このデータセットは第24太陽活動周期の初期から、第25周期の初期までの観測データを網羅している。この時間範囲には、太陽活動の活発性が高いフェーズと低いフェーズの両方が含まれている。従って、このデータセットは太陽活動の活発性に依存しない可能性が高く、その汎化能力に対する期待が一定程度裏付けられる。

\begin{table}[h]
    \centering
    \begin{tabular}{|c|c|c|}
    \hline
    実験 & 実験1 & 実験2 \\
    \hline\hline
    入力波長 & 211Å & 171Å,193Å,211Å \\
    \hline
    出力波長 & \multicolumn{2}{c|}{211Å} \\
    \hline
    総枚数 & 22000 & 66000 \\
    \hline
    セット数 & \multicolumn{2}{c|}{232} \\
    \hline
    セットごとの枚数 & \multicolumn{2}{c|}{入力12 → 出力12} \\
    \hline
    解像度 & \multicolumn{2}{c|}{512 * 512} \\
    \hline
    \end{tabular}
    \caption{各実験でのデータセット}
    \label{tab:my_label}
\end{table}

\subsection{破損画像の除去}
\subsection{画像の前処理}



