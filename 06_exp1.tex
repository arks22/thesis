\chapter{Motion Aware Unitを用いた1波長を入力とした紫外線像の全球時系列予測}
  \section{実験概要}
  実験1では入力、出力ともに211Åフィルターで得られたデータを利用した。これは 211Åフィルターで撮影された紫外線像が、コロナホールと活動領域といった、二つの太陽円盤上の大規模構造をバランスよく明瞭に表現し、本研究のモデルの効果検証に適していると考えたためである。
  \section{学習の推移}
  \section{実験結果}
    \subsection{全球での評価}
      \subsubsection{平均輝度とその誤差}
      \subsubsection{画像類似度}
      \subsubsection{単純差動回転モデルとの比較}
      \subsection{経度依存性の評価}
        \subsubsection{平均輝度とその誤差}
        \subsubsection{単純差動回転モデルとの比較}
    \subsection{東側リムから出現する活動領域に対する視覚的評価}
  \section{考察}
