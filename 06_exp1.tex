\chapter{Motion Aware Unitを用いた1波長を入力とした紫外線像の全球時系列予測}
  \section{実験概要}
  実験1では入力、出力ともに211Åフィルターで得られたデータを利用した。これは 211Åフィルターで撮影された紫外線像が、コロナホールと活動領域といった、二つの太陽円盤上の大規模構造をバランスよく明瞭に表現し、本研究のモデルの効果検証に適していると考えたためである。
  \section{実験設定}
    各ハイパーパラメータの設定を表\ref{tab:hyperparameters1}に示す。
    \begin{table}[h]
      \centering
      \begin{tabular}{|c|c|}
      \hline
      ハイパーパラメータ & 値 \\
      \hline\hline
      バッチサイズ & 4 \\
      \hline
      エポック数 & 100 \\
      \hline
      学習率 & 0.0005 \\
      \hline
      損失関数a & MSE \\
      \hline
      カーネルサイズ & (5, 5) \\
      \hline
      MAU Cell数 & 16 \\
      \hline
      \end{tabular}
      \caption{実験1でのハイパーパラメータ}
      \label{tab:hyperparameters1}
    \end{table}


  \section{学習の推移}
  \section{実験結果}
    \subsection{全球での評価}
      \subsubsection{平均輝度とその誤差}
      \subsubsection{画像類似度}
      \subsubsection{単純差動回転モデルとの比較}
      \subsection{経度依存性の評価}
        \subsubsection{平均輝度とその誤差}
        \subsubsection{単純差動回転モデルとの比較}
    \subsection{東側リムから出現する活動領域に対する視覚的評価}
  \section{考察}
