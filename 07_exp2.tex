\chapter{Motion Aware Unitを用いた3波長を入力とした紫外線像の全球時系列予測}
  \section{実験概要}
  実験2では入力に171Å、193Åフィルターで得られたデータを追加で利用した。これらの波長を追加することで、より広範な温度帯に渡る太陽活動をモデルが学習することを期待している。
  \section{学習の推移}
  \section{実験結果}
    \subsection{全球での評価}
      \subsubsection{平均輝度とその誤差}
      \subsubsection{画像類似度}
      \subsubsection{単純差動回転モデルとの比較}
      \subsection{経度依存性の評価}
        \subsubsection{平均輝度とその誤差}
        \subsubsection{単純差動回転モデルとの比較}
    \subsection{東側リムから出現する活動領域に対する視覚的評価}
  \section{考察}
