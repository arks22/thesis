\chapter{議論}

\section{全体的な考察}
  未知の太陽全球画像を予測することは、既存の予測モデルの予測能力を拡張し、専門家にとっても有用な情報源を提供する可能性があるため、より早期の宇宙天気予報の実現において有用である。
  本研究では、深層学習を用いた動画予測手法を用いて、SDO / AIAの時系列画像から、48時間以内の全球紫外線画像を生成することを目的とした。

  はじめに、AIAの211\AA フィルターから得られた時系列全球データを入力とし、MAUを用いて48時間以内の4時間ごとの全球紫外線画像を生成するモデルを構築した。
  このモデルを用いた実験では、生成された画像に対して、全球、経度ごと、さらに東側外縁部における輝度強度の再現性を評価した。
  さらに、単純作動回転モデルとの比較を行い、既存のシミュレーションモデルに対する性能を評価した。
  この実験では、MAUは各評価指標のほとんど全てにおいて単純差動回転モデルを上回り、また時空間的なロバスト性を持つことを確認することができた。
  
  次に、AIAの211\AA, 193\AA, 171\AA フィルターから得られた時系列全球データを入力とし、MAUを用いて48時間以内の4時間ごとの全球紫外線画像を生成するモデルを構築し、同じく評価を行った。
  この実験では、ほとんどの評価指標において、実験1の結果を下回るか、同等の結果となった。
  これは、現在使用してるMAUのアーキテクチャやモデルの深さでは、3波長の入力に対して十分な学習を行うことができないことが原因と考えられる。
  
  以上の結果から、本研究で構築したMAUを用いた動画予測モデルは、既存の予測モデルの予測能力を拡張することができることが示された。
  特に、以下の点については、本研究で用いた動画予測モデルの注目すべき強みであると言える。

  \paragraph{空間的ロバスト性}
    経度ごと評価や、東側外縁部における評価において、MAUは低い誤差を維持し、単純差動回転モデルよりも優れた性能を示した。
    東側外縁部以外の領域では、その面の角度とそれによる歪みの有無にかかわらず、MAUはほとんど同じ精度で予測を行うことができる。
    さらに、常に新しい面が観測される東側外縁部においても、MAUは間接的な情報を利用して予測を行ことができると考えられ、高い学習能力を持つことが示唆される。
    
    この高い学習性能とロバスト性は、全球を偏りなくバランスよく再現できるということであり、動画予測モデルの重要な特徴である。
    太陽表面での現象は、その位置により、地球への影響の程度が異なるため、全球の情報を正確に捉えることは重要である。
    そのような課題に対して、全球の広範囲にわたって正確に予測できるこのモデルは、宇宙天気予報において有用であると考えられる。
  
  \paragraph{時間的ロバスト性と確率的予測}
    動画予測モデルは、ほとんどの評価指標において、時間経過に伴う性能の低下が、単純差動回転モデルよりも緩やかであった。
    これは、深層学習の確率的モデリングの特徴と、太陽という複雑な系の相性が良いことが要因に考えられる。
  
  \paragraph{高速な予測}
    本研究で用いた動画予測モデルであるMAUは、その学習の完了に10時間単位の計算時間と高性能なGPUを要求するが、学習済みモデルによるテストデータに対する予測は数秒で完了する。
    これは、スーパーコンピュータレベルの計算リソースを必要とする物理シミュレーションモデルと比較して非常に高速であり低コストである。
    この点は、迅速な予測が求められる宇宙天気予報において重要であり、動画予測モデルの高い有効性を示すものである。

\section{今後の展望と課題}
  本実験の結果は、宇宙天気予報における多くの新しいアプローチの可能性を示唆するものであると言える。
  下に示すようなさらなる動画予測モデルの改良や、実際の宇宙天気予報モデルへの直接的な応用により、本研究の成果をさらに発展させることができると考えられる。
  
  \subsection{異なるサンプリング間隔での予測}
    本研究では、4時間おきのデータを入力として48時間以内の予測を行った。
    これは、数日単位での全球紫外線画像の予測を目的としているためであるが、さらに高い時間分解能での短い時間スケールでの予測や、逆に長い時間スケールでの予測も有用であると考えられる。

    短い時間スケールでは、活動領域などに限定した予測を行うことなどが考えられる。
    例えば、フレアの発生を予測するために、活動領域に予測範囲を限定し、より高いサンプリング間隔での予測を行うことが考えられる。
    フレアの発生は非常に複雑な現象であるため、本研究の予測モデルで十分な精度で予測できるかは不明である。
    しかし、後述するモデル変更などのアプローチを行うことで、予測の精度を向上させる可能性がある。

    長い時間スケールでは、コロナホールの形状の変化の予測などが考えられる。
    コロナホールは全球で観測される中でも大規模な構造であり、その形状の変化は、活動領域などに比べてゆっくりとした時間スケールで起こる。
    ある特定のコロナホールに対して、自転周期程度の時間をサンプリング間隔として予測を行うことで、数ヶ月先までのコロナホールの形状を予測することができる可能性がある。
    このような予測は、宇宙天気予報において、コロナホールによる高速太陽風の到達を予測するために有用であると考えられる。

  \subsection{より表現力の高い動画予測モデルによる予測}
    本研究では、CNNと再帰的ニューラルネットワークを組み合わせた動画予測モデルであるMAUを用いて予測を行った。
    しかし、近年の動画予測モデルの研究では、より表現力や精度の高いモデルが提案されている。
    特に、\citex{dosovitskiy2020image}らにより発表された、Transformerをベースとした画像処理モデルであるVision Transformer (ViT) の登場以降、Transformerをアーキテクチャの中核に置いた動画予測モデルの研究が注目を集めている (\citex{li2022efficient}, \citex{tang2023swinlstm} )。
    Transformerは、その注意構造から、解釈可能性の高いモデルとしても注目されており、モデルの予測の理由を解析することができる。
    それにより、現象の予測だけでなく、その現象のダイナミクスの解明にも役立つ可能性がある。

  \subsection{異なる観測データでの予測}
    本研究では、SDO / AIAで観測される太陽の遷移層からコロナの領域を捉える全球紫外線像を入力として予測を行った。
    本研究では、少なくともAIA 211\AA フィルターのデータを入力とした場合、有効な予測結果を得ることができた。
    さらなる宇宙天気予報への応用と拡張として、他の波長で観測される紫外線画像や、磁場データなどを入力とした予測を行うことが考えられる。
    例えば、SDO / HMIで観測される磁場データは、フレア予測において最も重要なデータの一つであり、これを予測対象とすることは、より直接的な宇宙天気予報への応用となる。
    また、黒点の成長予測など、より難しい予測への挑戦も有用である。

    このように、動画予測モデルは用いた予測は、太陽における多くのイベント、現象に対する汎用性があり、本研究で示された可能性はまだその一部に過ぎないと考えられる。

  \subsection{実際の宇宙天気予測モデルへの応用}
    本研究では、実際の宇宙天気予報モデルの予測能力の拡張や、専門家による宇宙天気予報への情報源の提供を将来的な目標としつつ、その前段として、輝度強度の再現度を評価することで、動画予測モデルの有効性を検証した。
    その評価の結果、深層学習を用いた動画予測モデルは、目的とする全球紫外線像を精度よく再現した。

    実際の宇宙天気予報モデルへ入力データとし、その予測性能が拡張可能であるかどうかは、より詳細で実際的な評価が必要である。
    今後、そのような評価を行うことで、本研究の成果を実際の宇宙天気予報モデルへの応用に繋げることができると考えられる。
  
