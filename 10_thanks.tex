\chapter*{謝辞}
  本研究は、筆者が新潟大学大学院自然科研究科に在学中に、飯田研究室において、飯田佑輔准教授のご指導のもの行われたものです。
  本研究のテーマは、宇宙天気の分野でも挑戦的なものであり、動画予測という手法もまだまだ新しい手法です。
  最初期の段階では、私の動画予測という手法を使いたいという動機から決定したテーマで、どのようなデータでどのような予測を行うか、どのような評価を行うかなど、多くの課題がありました。
  そのような中で、飯田准教授は、私の研究の方向性を見極め、的確なアドバイスを与えてくださいました。
  紫外線像での予測を行うこと、輝度強度の誤差率を評価すること、単純差動回転モデルとの比較を行うことなど、本研究の成果を大きく左右する重要なアドバイスを与えてくださいました。
  また、天文学会やJpGU、SGEPSS、Hinode-16 / IRIS-13 meetingなど、国際学会を含む多くの学会において、私の研究の発表の機会を与えてくださり、非常に多くのサポートをしてくださいました。
  学会での発表はプレッシャーも大きいものでしたが、他の学生や機関の研究者の方との交流は非常に有意義なものであり、モチベーションになりました。
  こうした飯田准教授のご指導のおかげで、本研究を進めることができました。
  この場を借りて、深く感謝申し上げます。
  
  国立研究開発法人情報通信研究機構 (NICT) の西塚直人様には、本研究の実施にあたり、多くの助言をいただきました。
  特に、実際に運用されている深層学習モデルのDeep Flare Netの開発者として、動画予測を用いた宇宙天気予報の可能性について、多くの示唆をいただきました。

  また、飯田研究室の先輩方、同期の方々には、研究の進め方や、学会発表の仕方など、多くのアドバイスをいただきました。
  特に、JpGUでの発表においては、初めての国際学会での英語発表であり、非常に緊張しましたが、ホテルで同室であった佐藤くんとは、同じ境遇であることを励まし合い、発表に臨むことができました。
  あの数日は、3年間の研究生活の中でも、最も濃厚な時間だったと思います。
  
  最後に、経済的にも精神的にも支えてくださった家族、親戚に心より感謝します。

  多くの方々のご協力のおかげで、本研究を進めることができました。
  この場を借りて、改めて深く感謝申し上げます。