% main.tex

\documentclass[a4paper,11pt]{jsreport}
\usepackage{pxrubrica} % ルビ(ふりがな)を使用する場合
\usepackage[dvipdfmx]{graphicx} % 画像の挿入のためのパッケージ
\usepackage{subcaption} % サブキャプションを扱うためのパッケージ
\usepackage{url} % URLの挿入のためのパッケージ
\usepackage{multirow}
\usepackage{comment}
\usepackage{array}
\usepackage{parskip}
\usepackage{otf}
\usepackage{amsmath}
\usepackage{algorithm}
\usepackage{algpseudocode}
\usepackage{fancyhdr}
\usepackage[numbers]{natbib}

\newcommand{\citex}[1]{\citeauthor{#1} (\citeyear{#1})}

\pagestyle{fancy}
\fancyhf{} % ヘッダーとフッターをクリア
\fancyhead[L]{\nouppercase{\rightmark}} % ヘッダー左にセクション名を表示
\fancyfoot[C]{\thepage} % フッター中央にページ番号を表示

\setcounter{secnumdepth}{3} % タイトルにどこまの深さまで番号をつけるか
\setcounter{tocdepth}{3}    % 目次にどこまで表示するか

\DeclareFontShape{JY1}{mc}{m}{it}{<->ssub*mc/m/n}{}
\DeclareFontShape{JY1}{mc}{m}{sl}{<->ssub*mc/m/n}{}
\DeclareFontShape{JY1}{mc}{m}{sc}{<->ssub*mc/m/n}{}
\DeclareFontShape{JY1}{hgt}{m}{it}{<->ssub*hgt/m/n}{}
\DeclareFontShape{JY1}{hgt}{m}{sl}{<->ssub*hgt/m/n}{}
\DeclareFontShape{JY1}{mc}{bx}{it}{<->ssub*gt/m/n}{}
\DeclareFontShape{JY1}{mc}{bx}{sl}{<->ssub*gt/m/n}{}
\DeclareFontShape{JT1}{mc}{m}{it}{<->ssub*mc/m/n}{}
\DeclareFontShape{JT1}{mc}{m}{sl}{<->ssub*mc/m/n}{}
\DeclareFontShape{JT1}{mc}{m}{sc}{<->ssub*mc/m/n}{}
\DeclareFontShape{JT1}{hgt}{m}{it}{<->ssub*hgt/m/n}{}
\DeclareFontShape{JT1}{hgt}{m}{sl}{<->ssub*hgt/m/n}{}
\DeclareFontShape{JT1}{mc}{bx}{it}{<->ssub*gt/m/n}{}
\DeclareFontShape{JT1}{mc}{bx}{sl}{<->ssub*gt/m/n}{}

\begin{document}

\thispagestyle{empty} % ヘッダーやフッターを表示させない

\begin{center}
  \leftline{2023 年度卒業}

    \vspace{2cm}
    
    {\Huge 修士論文}
    
    \vspace{2cm}
    
    {\LARGE 深層学習による動画予測手法を用いたSDO紫外線画像の全球時系列予測}
    
    \vspace{1cm}
    
    {\Large a Time-Series Prediction of SDO Ultraviolet Full-disk Images using a Video Prediction Method with Deep Learning}
    
    \vspace{3cm}
    
    \Large
    \begin{tabular}{|c|c|}
      \hline
      \multirow{3}{*}{所属} & 新潟大学 大学院自然科学研究科\\ &  電気情報工学専攻 情報工学コース \\ & 飯田研究室 \\
      \hline
      氏名 &  佐々木明良\\
      \hline
      学籍番号 & F22C017D\\
      \hline
    \end{tabular}
\end{center}

\clearpage % これにより新しいページに移動


\chapter*{概要}
  太陽活動に起因する激しい宇宙天気の変動は、地球上の電力網や衛星通信などの技術システムに深刻な影響を与える可能性がある。
  そのため、太陽活動の観測と宇宙天気の予測は、人類の社会活動を守る上で重要であり、その予測モデルの開発や、現象のメカニズムの研究は盛んに行われている。
  宇宙天気予報において、NASAのSolar Dynamics Observatory (SDO) をはじめとする観測衛星から得られる観測データが重要な役割を果たしている。
  これらのデータは予測モデルの入力や、専門家による予報の情報源として利用されている。
  
  近年の深層学習の進展により、宇宙天気予報においてもその応用が見られるようになった。
  深層学習モデルは迅速な予測を可能にし、確率論的なアプローチを通じて不確実性を考慮することができる。
  NICTによるDeep Flare Netや、NASAによるDeep Learning Geomagnetic Perturbationなど、深層学習を利用した予測モデルは実際に運用されており、宇宙天気予報においてその有用性が確認されている。
  
  宇宙天気による影響から重要な技術システムを保護するには、早期かつ正確な予測が必要である。
  しかし、シミュレーションによる予測モデルや深層学習による予測モデルは、多くの場合、数時間から数日後までの予測に焦点を当てている。
  さらに長い時間範囲での予測が理想であるが、計算資源の制約や、太陽の複雑なシステムに対するモデルの予測能力の限界などの理由により、その実現は困難である。

  この課題の克服のために、未知の太陽画像を予測、生成し、既存の予測モデルの入力データとして利用することが考えられる。
  そのような背景から、本研究では、深層学習を用いた、未来の太陽画像の生成という新しいアプローチを提案する。
  具体的には、動画予測という技術を用いて、SDO / AIAから得られる全球紫外線画像の予測を試みる。
  本研究により、未知の太陽画像を精度よく再現することができれば、既存の予測モデルの入力データとして利用することで予測可能な時間範囲を拡張することができ、専門家によるより早期の宇宙天気予報の実現に貢献することができると考えられる。

  動画予測とは、動画の一部を入力として、それに続く未来の動画を予測するタスクである。
  CNNを中心とした画像処理技術の進展と、LSTMを中心とした時系列データの処理技術の進展、およびその融合により、近年注目を集めている。
  
  はじめに、Motion-Aware Unitと呼ばれる動画予測モデルを用いて、SDO / AIAの211\AA フィルターから得られた時系列全球データを入力とし、48時間以内の4時間ごとの全球紫外線画像を生成するモデルを構築した。
  生成された予測画像に対し、全球、経度ごと、さらに東側外縁部における輝度強度の再現性を評価した。
  さらに、より詳細な性能評価と比較のために、単純作動回転モデルとの比較を行った。
  全球での平均輝度強度の絶対誤差は、48時間後の予測で3.67\% であり、単純差動回転モデルの絶対誤差である10.2\% よりも低く、高い再現性を示した。
  経度ごとの評価でも、すべての分割セクターにおいて、単純差動回転モデルよりも高い精度を示した。
  また、東側外縁部においても、良好な性能を示し、動画予測モデルの高い学習能力が確認された。
  さらに、動画予測モデルが予測に要求する時間は数秒であり、非常に高速に予測画像を生成することができた。

  次に、さらなる性能向上を目的として、211 \AA フィルターの全球画像に加え、171 \AA フィルター、193 \AA フィルターの全球画像を追加し、3波長の入力から211 \AA フィルターの画像を予測するモデルを構築した。
  評価は前実験と同様に行った。
  この実験では、ほとんどの評価指標において、1波長入力での結果を下回るか、同等の結果となった。
  これは、現在使用してる動画予測モデルのアーキテクチャやモデルの深さでは、3波長の入力に対して十分な学習を行うことができないことが原因と考えられる。

  このような結果から、深層学習を用いた動画予測手法は、未知の太陽全球紫外線像の予測という課題において、有効な手法であることが示された。
  特に、高い時空間的ロバスト性を実現する学習能力や、予測の高速性は、実際の宇宙天気予報において求められる重要な条件である。
  今後、本研究の結果をもとにしたさらなるアプローチや改善手法の検討を行うことで、本研究の成果をさらに発展させることができると考えられる。



\tableofcontents

\chapter{序論}

\section{研究背景}
  \subsection{宇宙天気}
  宇宙天気とは、太陽活動に起因する宇宙環境の現象を指し、激しい宇宙天気の変動は地球の磁場や電離層、大気圏に影響を与える(図\ref{fig:space_weather_impact})。
  これは大規模なものであれば、地球上の電力網や衛星通信など、人間の技術システムに壊滅的被害をもたらす可能性がある。
  そのため、宇宙天気を予測することは、人間の技術システムを宇宙天気の影響から守るための重要な課題である。
  \begin{figure}[htbp]
    \centering
    \includegraphics[width=0.9\textwidth]{figures/spaceweather.jpg}
    \caption[宇宙天気擾乱と社会への影響]{宇宙天気擾乱の発生と社会への影響の概念図。出典: 国立情報通信研究所, 「宇宙天気予報とは」, \url{https://swc.nict.go.jp/knowledge/relation.html}(アクセス日: 2024年1月9日)}
    \label{fig:space_weather_impact}
  \end{figure}

  太陽の表面や大気における物理現象、特に太陽フレアやコロナ質量放出(CME)などの爆発的なイベントは、地球に到達する高エネルギー粒子や放射線の量を増加させ、宇宙天気に大きな影響を与える。
  宇宙天気を予報するためには、そういった太陽イベントを観測し、それらの観測データを解析することが必要である。
  実際の宇宙天気予報は、これらの観測データを入力としたシミュレーションモデルや機械学習モデルを常に実行し、さらに定期的に専門家による主観的な予報を行うことが多い。

  % 宇宙天天気予報のための観測
  % 観測の重要性
  % 観測をもとにモデルを実行したり主観的な分析をして数日先の予報を行う
  \subsection{観測データ}
  刻一刻と変化する太陽の状態から宇宙天気を予報するために、高精度な観測データは必要不可欠である。
  観測は、さまざまな観測機器を搭載した観測衛星に加え、世界各国に存在する地上の望遠鏡や電波望遠鏡などで行われる。
  Solar Dynamics Observatory (SDO) (\citex{pesnell2012solar})や、Solar and Heliospheric Observatory (SOHO) (\citex{domingo1995soho})、「ひので」 (\citex{kosugi2008hinode})などの太陽観測衛星からリアルタイムで提供される、太陽の表面や大気の観測データは非常に重要である。

  後述するDeep Flare Netなどに代表される太陽フレア予測モデルでは、SDOに搭載されたHelioseismic and Magnetic Imager (HMI) (\citex{scherrer2012helioseismic})や、Atmospheric Imaging Assembly (AIA) (\citex{lemen2012atmospheric})から得られる磁場データや紫外線データ(図\ref{fig:sdo_axmple})を入力として、数時間後から数日後までの太陽フレアの発生を予測する。
  Magnetohydrodynamics (MHD)シミュレーションモデルに代表される多くの数値シミュレーションモデルにおいても、それらの衛星から得られた太陽表面のデータを入力として予測を行う。
  例えば、\citex{shiota2016magnetohydrodynamic}, \citex{shiota2021susanoo}による太陽嵐予測システムSUSANOO-CMEは、実際に運用されている高性能なシミュレーションモデルであり、SOHOやSDOなどの複数の太陽観測衛星による得られたコロナグラフや極紫外線データを入力とする。
  これをもとに太陽風、CMEの伝搬をシミュレーションすることで、いち早く太陽嵐の到達時刻を予測することができる。
  \begin{figure}[htbp]
    \begin{subfigure}{0.5\textwidth}
      \centering
      \includegraphics[width=\textwidth]{figures/latest_512_0211.jpg}
    \end{subfigure}
    \begin{subfigure}{0.5\textwidth}
      \centering
      \includegraphics[width=\textwidth]{figures/latest_512_HMIB.jpg}
    \end{subfigure}
    \caption[SDO]{SDOに搭載されるAIAの211\AA フィルターによって観測される全球紫外線画像 (左)とHMIによって観測される磁場データ (右)。出典: NASA, 「SDO | Solar Dynamics Observatory」, \url{https://sdo.gsfc.nasa.gov/}(アクセス日: 2024年1月9日)}
    \label{fig:sdo_axmple}
  \end{figure}
  
  このように、現在の予測モデルにおいてはその入力データとして、観測衛星にとって得られるデータは非常に重要な役割を果たしている。

  \subsection{深層学習の活用}
    深層学習は、近年顕著な発展を遂げた機械学習の一分野である。
    効率的で高性能なモデルの登場、ハードウェアのデータ処理能力の向上、データセットの増加などがその発展を支えてきた。
    深層学習は、学術的な分野での成功のみならず、産業界においても、画像認識や自然言語処理などの分野での応用が進んでおり、情報化社会の現代においては欠かせない技術となっている。

    近年の宇宙天気予報においても、機械学習、特に深層学習アーキテクチャの基本とする予測モデルが多く開発されている (\citex{camporeale2019challenge} )。
    宇宙天気予報にとって、深層学習モデルの高速性と、不確実な現象への確率論的アプローチの有効性は特に重要な点である。
    深層学習は、蓄積された過去の大量のデータからパターンを学習するデータ駆動型のモデルであり、予測の実行は非常に高速である。
    大量の高性能なコンピューティングマシンを必要とするシミュレーションモデルより、はるかに小さい計算資源でリアルタイムでの予測が可能である。
    宇宙天気の予報は迅速に行う必要があるため、この高速性は深層学習モデルの大きな利点である。
    また、太陽フレアやコロナ質量放出などの宇宙天気現象の発生メカニズムは非常に複雑であり、完全にシミュレーションモデルで再現することは困難である。
    そのようなシステムに対して、深層学習モデルは確率論的なアプローチをとることができ、予測においてその不確実性を考慮することができる。

    宇宙天気予報における深層学習を用いた予測モデルの開発では、いくつかの注目すべき研究がある。
    \citex{nishizuka2018deep}によるDeep Flare Net (DeFN)は、太陽フレアの発生とその強度を予測するための深層学習モデルである。
    DeFNはSDOに搭載されたAIAやHMIから得られる紫外線データ、磁場データから作成された特徴量を用いて、24時間以内の太陽フレアの発生を高い精度で予測することができる。
    このモデルは予測において、注目している活動領域に対してフレアの発生とその強度を確率的に予報する。
    DeFNは、国立研究開発法人情報通信研究機構 (NICT)によって実際に運用されている。

    \citex{upendran2022global}によるDeep Lerning Geomagnetic perturbation (DAGGER)は、NASAにより開発された、衛星で観測された太陽風データをもとに、地上の地磁気摂動の予測に特化した深層学習モデルである。
    太陽フレアの爆発やCMEが発生し、地球に高エネルギー粒子が太陽風として到達すると、しばしば地磁気嵐とそれに伴う地磁気摂動が発生する。
    このような高強度の太陽風の飛来を早期に検出するために、Advanced Composition Explorer (ACE) (\citex{stone1998advanced})やWINDなどの衛星が存在する。
    これらの衛星は、地球と太陽の重力が釣り合うラグランジュ点に配置され、太陽風の強度や速度などを地球に到達する前に観測することができる。
    太陽風の速度は衛星が発する電波よりも遅く、地磁気嵐の予測においてはこの時間差をリードタイムとして利用することができる。
    しかし、高速太陽風においてはこのリードタイムは数十分程度であり、予測には高い迅速性が求められる。
    深層学習モデルであるDAGGERは、このような迅速性が求められる予測において、高い有効性を持つ。
    DAGGERが予測に要求する時間は1秒以下であり、太陽風到着までの時間的猶予がない状況でも、保護すべき技術システムに対して最大限の対応時間を与えることができる。
    
%%%%%%%%%%%%%%%%%%%%%%%%%%%%%%%%%%%%%%%%%%%%%%%%%%%%%%%%%%%%%%%%%%%%%%%%%%%%%%%%%%%%%%%%%%%%%%%%%%%%%%%%%%%%%%%%%%%%%%%%%%%%%%%%%%%%%%%%%%%%%%%%%%%%%%%%%
%%%%%%%%%%%%%%%%%%%%%%%%%%%%%%%%%%%%%%%%%%%%%%%%%%%%%%%%%%%%%%%%%%%%%%%%%%%%%%%%%%%%%%%%%%%%%%%%%%%%%%%%%%%%%%%%%%%%%%%%%%%%%%%%%%%%%%%%%%%%%%%%%%%%%%%%%

\section{問題定義}
% 「観測データ」、「深層学習の活用」、から繋げるイメージで。
% 多くのモデルや専門家による予報では、数時間後から数日後までの予報が多く、情報ソースはSDOなどの観測衛星に頼っている
% もし数日後のデータが得られれば迅速さに貢献し有用である
% 数日後のデータが得られれば有用であるが、シミュレーションモデルでの予測は計算時間などの制約で難しい、特に全球は難しい
宇宙天気予報の分野では、迅速かつ正確な予測が重要である。
現在利用されているシミュレーションモデルや深層学習モデルの多くは、数時間から数日後までの予報に焦点を当てている。
この時間範囲は、太陽の複雑でカオス的な性質を鑑みると、現在の予測モデルの現実的な限界であると考えられる。
しかし、重大な宇宙天気現象は、特に影響を受ける技術システムに対する対応策を講じる必要性があることを鑑みると、より早期に予測することが理想である。

そのために、数日後の太陽の姿を予測し、観測データとして入力とすることがアイディアとして考えられる。
それが可能であれば、既存の予測モデルのアーキテクチャを変更することなく、それらのモデルの予測可能な時間範囲をその分だけ拡張することができる。
また、専門家による主観的な予報においても、より早期に正確な予報が可能になることが期待される。
しかし、現在はそのような研究はほとんど行われておらず、その可能性は未知数である。
これにはいくつかの理由が考えられるが、太陽の時空間的特徴を正確に捉えるモデルを作成することが難しいことや、計算資源の制約などが挙げられる。
例えば、既存のMHDモデルでは、その結果として太陽の表面や大気のシミュレーションデータを得ることができるが、その計算には高い計算負荷と時間が必要であり、予測可能時間の拡大という目的で用いることは困難である。

このような問題の解決として、本研究では、深層学習による将来の太陽画像の生成という新しいアプローチを提案する。
動画予測は、深層学習の分野において近年注目されているタスクであり、その応用例は多岐にわたる。
動画予測は、動画の一部を入力として、それに続く未来の動画を予測するタスクである。
データセットとして利用できる動画や、その他の二次元の時系列データがあれば学習可能なため、動画だけではなく、気象予報や渋滞予測など、様々な分野での応用が期待されている。

本研究では、深層学習による動画予測手法を用いて、特に既存の予測モデルの入力データとして重要な、SDO / AIAから得られる全球紫外線データを予測する。
深層学習による高速な予測能力と、動画予測モデルの長期的な依存関係を学習する能力を活用し、数日後の全球紫外線画像を生成することで、より早期の宇宙天気予報の実現に貢献することを将来的な目的とする。

%%%%%%%%%%%%%%%%%%%%%%%%%%%%%%%%%%%%%%%%%%%%%%%%%%%%%%%%%%%%%%%%%%%%%%%%%%%%%%%%%%%%%%%%%%%%%%%%%%%%%%%%%%%%%%%%%%%%%%%%%%%%%%%%%%%%%%%%%%%%%%%%%%%%%%%%%
%%%%%%%%%%%%%%%%%%%%%%%%%%%%%%%%%%%%%%%%%%%%%%%%%%%%%%%%%%%%%%%%%%%%%%%%%%%%%%%%%%%%%%%%%%%%%%%%%%%%%%%%%%%%%%%%%%%%%%%%%%%%%%%%%%%%%%%%%%%%%%%%%%%%%%%%%

\section{本研究の目的}
  本研究では、深層学習による動画予測手法を用いて、SDO / AIAの時系列画像から、48時間以内の全球紫外線画像を生成することを目的とする。
  さらに、生成された画像に対して、既存の予測モデルの入力データとして用いられる輝度強度の再現性を評価することで、提案手法の有効性を検証する。
  有効性は、経度ごとでの評価や、全球のうち予測が難しいと思われる局所的な領域に対する評価、さらにシンプルなシミュレーションモデルとの比較など、さまざまな条件下で行い、モデルの時空間的ロバスト性を評価する。
  また、入力として用いる波長を変化させ、太陽表面の複雑な相互作用をモデルがどの程度モデリングできるかを評価し、その結果を考察する。

%%%%%%%%%%%%%%%%%%%%%%%%%%%%%%%%%%%%%%%%%%%%%%%%%%%%%%%%%%%%%%%%%%%%%%%%%%%%%%%%%%%%%%%%%%%%%%%%%%%%%%%%%%%%%%%%%%%%%%%%%%%%%%%%%%%%%%%%%%%%%%%%%%%%%%%%%
%%%%%%%%%%%%%%%%%%%%%%%%%%%%%%%%%%%%%%%%%%%%%%%%%%%%%%%%%%%%%%%%%%%%%%%%%%%%%%%%%%%%%%%%%%%%%%%%%%%%%%%%%%%%%%%%%%%%%%%%%%%%%%%%%%%%%%%%%%%%%%%%%%%%%%%%%

\section{本論文の構成}
  \paragraph{第2章 動画予測}
    動画予測という深層学習のタスクについて、その基本概念と基礎技術、および本研究で用いる動画予測モデルについて述べる。
  \paragraph{第3章 データ}
    本研究で用いるデータセットについて述べる。特に、各波長データの特性、およびデータセットの構築方法について述べる。
  \paragraph{第4章 Motion-Aware Unitを用いた1波長を入力とした紫外線像の全球時系列予測}
    本研究で提案する動画予測モデルを用いて、1波長を入力とした紫外線画像の全球時系列予測を行う。その結果をさまざまな条件下で評価し、モデルの有効性を検証する。
  \paragraph{第5章 Motion-Aware Unitを用いた3波長を入力とした紫外線像の全球時系列予測}
    本研究で提案する動画予測モデルを用いて、3波長を入力とした紫外線画像の全球時系列予測を行う。1波長の場合と同様に、その結果をさまざまな条件下で評価し、性能の変化とその原因を考察する。
  \paragraph{第6章 議論}
    本研究の実験全体を通した考察と、今後の展望及び課題について述べる。
  \paragraph{第7章 結論}
    本研究の結果、議論から得られた結論をまとめる。


\begin{comment}
    
そこで、まだ観測されていない未来の紫外線画像を予測、生成することで、より早期の宇宙天気予報の実現に貢献できるのではないかと考えた。

近年、深層学習技術の発展により、「動画予測(Video Prediction)」と呼ばれる技術が注目されている。動画予測とは、動画の一部を入力として、それに続く未来の動画を予測するタスクである。
動画予測を行う深層学習モデルのアーキテクチャの多くは、Long Short-Term Memory(LSTM)などの再帰的ニューラルネットワーク(RNN)のアーキテクチャを基本とする。さらに、特徴量抽出および伝播にConvolutional Neural Network(CNN)を用いることで空間的な特徴を時系列にわたってとらえ、Decoderモデルによって動画を生成する。
このような動画予測モデルは、Conv-LSTMの登場で初めて提案されて以降、様々なモデルが提案されており、自動運転や天気予報など、様々な分野での応用が期待されている。

本研究では、動画予測モデルを用いて、数日後の紫外線画像を予測、生成することを目的とする。
Deep Flare Netなどの多くの深層学習を用いた宇宙天気予報モデル、またそれらを用いた人間による主観的な宇宙天気予報は、現在の観測情報を用いて、数時間後から数日後までの宇宙天気を予測するものが多い。
それらの情報源として、数日後の高精度な紫外線画像を生成することができれば、より早期の宇宙天気予報の実現に貢献できるのではないかと考えた。

そのような数日後の紫外線像の生成のために、本研究ではMotion-Aware Unit(MAU)と呼ばれる動画予測モデルを用いる。
MAUは、RNNやCNNを用いた基本的な動画予測モデルを基本としつつ、各時間時点における画像の処理にMAU Cellと呼ばれるモジュールを多層的に積み重ねたアーキテクチャを採用している。
MAU CellはAttentionと呼ばれる機構を持ち、長期的な依存関係を適切に学習する能力を持つ。また、従来の動画予測モデルで提案されてきたEncoder-Decoderモデルや、メモリフローの改善など、様々な改良が加えられており、多くの動画予測モデルの中でも要求する計算量に対して高い予測精度を達成している。

本研究では、MAUを用いて、数日後の紫外線画像を予測、生成することを目的とするが、その評価には主に輝度強度の再現性を用いる。
これは、宇宙天気予報モデルの多くではその特徴量として輝度強度を用いていること由来する。

+シミュレーションモデルでの予測
+ 宇宙天気分野での深層学習の隆盛
+ 宇宙天気予報における紫外線画像の重要性(NICT)
+ 動画予測の応用例
+ 宇宙天気のインパクト https://link.springer.com/article/10.1186/s40623-021-01420-5
+ 過去のデータを用いたデータ駆動型のMLモデルの宇宙天気予報にける有用性

1 宇宙天気の=を予測することは大事
2 そのために観測してる
3 宇宙天気分野ではデータ駆動型の深層学習モデルが適していて、たくさん行われている
4 多くのモデルや専門家による予報では、数時間後から数日後までの予報が多く、情報ソースはSDOなどの観測衛星に頼っている
5 数日後のデータが得られれば有用であるが、シミュレーションモデルでの予測は計算時間などの制約で難しい、特に全球は
6 そこで、動画予測モデルを用いて、数日後の紫外線画像を予測、生成することを目的とする    


\end{comment}
\chapter{動画予測}

\section{基本的な動画予測技術}

\section{Motion Aware Unit}

\chapter{データ} 

\section{SDO / AIA}
モデルの学習及び評価データとして、NASAのSolar Dynamic Observatory(SDO)\cite{pesnell2012solar}のAtmospheric Imaging Assembly(AIA)\cite{lemen2012atmospheric}で撮影された紫外線観測データを用いた。

SDOはNASAのLiving With a Star(LWS)プログラムの一つとして2010年2月に打ち上げられた太陽観測衛星である。
AIA、Helioseismic and Magnetic Imager(HMI)、Extreme Ultraviolet Variability Experiment(EVE)などの高い空間解像度、時間分解能を持つ観測機器を搭載している。
その観測データを用いることにより、太陽物理学、宇宙天気、また地球環境に関する理解や洞察を深めることが期待されている。
AIAは主に太陽大気を観測する観測機器であり、4つの望遠鏡で構成されている。
4096×4096、約1.5秒角の空間解像度をもち、12秒ごとに太陽全球の画像を観測している。
また、7つの極紫外線フィルターと、2つの紫外線フィルター、および1つの可視光フィルターを持ち、広範な温度帯で太陽大気を観察することを可能にしている。

これらのデータはJoint Science Operations Center(JSOC) によって提供されており、Pythonの太陽物理学を支援するライブラリであるSunpyを用いてダウンロードすることができる。


\subsection{AIA 211Å}
観測されるイオン、温度範囲、主な観測対象(活動領域、CHなど)

\subsection{AIA 193Å}
\subsection{AIA 171Å}

\begin{figure}[h]
    \centering
    \includegraphics[width=0.8\textwidth]{figures/latest_256_0211.jpg}
    \caption{SDO/AIAの211Åフィルターで撮影された太陽全球紫外線像。強調のために紫色に色付けされている。球面の中上部から中下部には明るく輝く活動領域が見られ、左下部に暗くコロナホールが観測できる。}
    \label{fig:sample_aia211}
\end{figure}
\section{前処理}


本研究で用いるデータセットには、SDO/AIA望遠鏡のデータが提供されている2010年5月から、2022年10月までのデータが含まれている。\\
この期間に存在するデータから、4時間ごとにデータを抽出し、各波長ごとに約22000枚(正確な方がいい?)をデータセットに含んでいる。これらのデータを、24枚の画像を1セットとして分割する。各セットは24枚の時系列に並んだ画像で構成され、太陽の時間的依存性および空間的情報を同時に捉えている。24枚のうち、前半の12枚、すなわち44時間後までを入力データ、後半の12枚、すなわち48時間後から92時間後までを出力データとして扱う。学習の際は、前半の12枚に対して後半の12枚を教師データとして扱い、テストの際は12枚の画像データに続くモデルにとって未知の12枚を再現できるか検証する。


このデータセットは第24太陽活動周期の初期から、第25周期の初期までの観測データを網羅している。この時間範囲には、太陽活動の活発性が高いフェーズと低いフェーズの両方が含まれている。従って、このデータセットは太陽活動の活発性に依存しない可能性が高く、その汎化能力に対する期待が一定程度裏付けられる。



\begin{table}[h]
    \centering
    \begin{tabular}{|c|c|c|}
    \hline
    実験 & 実験1 & 実験2 \\
    \hline\hline
    入力波長 & 211Å & 171Å,193Å,211Å \\
    \hline
    出力波長 & \multicolumn{2}{c|}{211Å} \\
    \hline
    総枚数 & 22000 & 66000 \\
    \hline
    セット数 & \multicolumn{2}{c|}{232} \\
    \hline
    セットごとの枚数 & \multicolumn{2}{c|}{入力12 → 出力12} \\
    \hline
    解像度 & \multicolumn{2}{c|}{512 * 512} \\
    \hline
    \end{tabular}
    \caption{各実験でのデータセット}
    \label{tab:my_label}
\end{table}

\subsection{不正な画像の除去}
SDO/AIA望遠鏡で撮影された全球画像には、露光時間が他の画像より極端に低い、画像内に太陽全体を捉えていない、などの不正な画像が含まれている。
確認することができた主な不正な画像を図\ref{fig:bad_aia_samples}に示す。

\begin{figure}[htbp]
    \centering
    \begin{subfigure}[b]{0.48\textwidth}
        \includegraphics[width=\textwidth]{figures/bad_sample0.png}
        \caption{短い露光時間により、極端に暗い画像。}
    \end {subfigure}
    \hfill
    \begin{subfigure}[b]{0.48\textwidth}
        \includegraphics[width=\textwidth]{figures/bad_sample1.jpg}
        \caption{太陽が画像の中心にない画像。}
    \end {subfigure}
    \begin{subfigure}[b]{0.48\textwidth}
        \includegraphics[width=\textwidth]{figures/bad_sample2.jpg}
        \caption{衛星が回転しており、正しい角度で太陽が撮影されていない画像。活動領域の少ない左下部と右上部が極である。}
    \end {subfigure}
    \hfill
    \begin{subfigure}[b]{0.48\textwidth}
        \includegraphics[width=\textwidth]{figures/bad_sample1.jpg}
        \caption{日蝕により太陽が隠されている画像}
    \end {subfigure}
    \caption{SDO/AIAにより観測された不正な画像の例}
    \label{fig:bad_aia_samples}
\end{figure}

これらの画像は、モデルの学習に悪影響を及ぼす可能性があるため、データセットから除去した。
機械学習のタスクによっては、十分なデータセットがあれば、モデルが不正な画像に対する頑健性を獲得し、不正な画像がデータセットに含まれていても、学習結果にあまり大きな影響を与えない場合がある。
しかし、本研究で行う動画予測は、データセットに含まれる画像がそのまま教師データとなるため、不正な画像は損失関数の計算、またはモデルの評価に大きな影響を与えるため、慎重に除去する必要がある。
データの除去には、FITSファイルのヘッダーに記録された各キーワードの値に対して閾値を設定して判定したのち、numpyによる輪郭検出を用いた月蝕判定関数により不正な画像を排除した。この流れを図に示す。
月蝕判定関数は、〜〜〜〜〜

\subsection{スケーリングと正規化}

\subsection{データセットの分割}

このようにして作成されたデータセットは、約1000セットになり、これを学習用データセットに約800セット、検証データセットに50セット,テストデータセットに50セットというように分割した。



\chapter{Motion-Aware Unitを用いた1波長を入力とした紫外線像の全球時系列予測}
  \section{実験概要}
  この実験では入力、出力ともに211Åフィルターで得られたデータを利用した。
  これは 211Åフィルターで撮影された紫外線像が、コロナホールと活動領域といった、二つの太陽円盤上の大規模構造をバランスよく明瞭に表現し、本研究のモデルの効果検証に適していると考えたためである。
  モデルにはMotion-Aware Unitを用い、1波長のデータを入力として、全球の時系列予測を行った。
  この実験の概要を図\ref{fig:exp1_overview}に示す。

  \begin{figure}[htbp]
    \centering
    \includegraphics[width=\textwidth]{figures/exp1/exp1_concept.jpg}
    \caption{実験の概念図。モデルにはMotion-Aware Unitを用い、1波長のデータを入力として、全球の時系列予測を行った。}
    \label{fig:exp1_overview}
  \end{figure}

  \section{実験設定}
    各ハイパーパラメータの設定を表\ref{tab:exp1_hyperparameters}に示す。
    バッチサイズは実験的に決定し、最も安定的に最終的に良好な精度を達成できた値を採用した。
    また、エポック数は100とした。学習率は0.0005とした。MAU Cell数は、 \cite{chang2021mau} の実験設定を参考に、16とした。
    学習時間の短縮およびメモリ使用量の削減のため、学習時にはAutomatic Mixed Precision (AMP)(\cite{micikevicius2017mixed})を用いた。
    これは、単精度浮動小数点演算と半精度浮動小数点演算を適切に混在させることで、モデル性能をほとんど落とさずに計算資源を節約し学習を高速化する手法である。
    GPUはNVIDIA RTX A6000を用いた。
    \begin{table}[htbp]
      \centering
      \begin{tabular}{lc}
      \hline
      ハイパーパラメータ & 値 \\
      \hline\hline
      バッチサイズ & 4 \\
      \hline
      エポック数 & 100 \\
      \hline
      学習率 & 0.0005 \\
      \hline
      損失関数 & MSE \\
      \hline
      チャンネル & 1 \\
      \hline
      カーネルサイズ & (5, 5) \\
      \hline
      MAU Cell数 & 16 \\
      \hline
      \end{tabular}
      \caption{本実験でのハイパーパラメータ設定}
      \label{tab:exp1_hyperparameters}
    \end{table}

  \section{学習の推移}
  学習は図\ref{fig:exp1_learn_progress}のように推移した。
  学習損失は全体的に安定して推移し、検証損失は時折急激に値が増加しているが、全体的には減少している。
  学習の完了までには約12時間を要した。
  \begin{figure}[htbp]
    \centering
    \includegraphics[width=0.8\textwidth]{figures/exp1/loss.png}
    \caption{本実験での、学習データ、検証データでの損失関数の推移。学習の損失は安定している。検証の損失は振動しながら減少している。}
    \label{fig:exp1_learn_progress}
  \end{figure}

  \section{実験結果}
    図\ref{fig:exp1_gt}および図\ref{fig:exp1_pd}に、この実験での出力例を示す。
    これは学習データに含まれない期間のテストデータである。
    \begin{figure}[htbp]
      \centering
      %\vspace*{-4cm} % 上の余白を調整
      \includegraphics[width=0.95\textwidth]{figures/exp1/gt.png}
      \caption{実際の観測画像の例。2022年10月28日0時から2022年11月1日20時までの期間から4時間毎にサンプリングされている。このt=0からt=11までをモデルに入力データとして渡している。モデルはその入力データを元に、t=12からt=23の12枚の画像を予測する。t=12以降の実際の観測画像はモデルに渡されない。}
      %\vspace{-1cm} % 下の余白を調整
      \label{fig:exp1_gt}
    \end{figure}
    \begin{figure}[htbp]
      \centering
      \includegraphics[width=0.95\textwidth]{figures/exp1/pd.png}
      \caption{MAUによる予測画像。対応するタイムステップtの観測画像(図\ref{fig:exp1_gt})と比較することでモデルの再現度を視覚的に評価することができる。大規模な構造は概ね実際の観測画像と合致している。モデルの特性により、時間経過とともに少しずつ予測が不安定になり、ぼやけた見た目になる。}
      \label{fig:exp1_pd}
    \end{figure}
    モデルの出力は、視覚的には実際の観測画像と概ね合致しており、特に自転による大規模構造の移動といった顕著な時間的特徴は再現できていることがわかる。

    動画予測の精度を評価するために、太陽の輝度強度の再現性を定量的、またまたは視覚的に評価する。
    これは、\citex{nishizuka2018deep} や〜〜〜などの、太陽画像から太陽イベントを予測する先行研究では、その画像中の輝度強度を主要な特徴量として採用していることに基づく。
    この輝度強度の再現性の評価を、さまざまな条件下で行った。はじめに全球での評価を行い、次に経度依存性の評価を行った。最後に、東側リムから出現する活動領域に対する視覚的評価を行った。

% *************************************************************************************************************

    \subsection{全球での評価}
      はじめに全球での評価を行った。
      この評価では、まず輝度強度の平均値と実際の平均値との誤差、構造的類似度(Structual Similarity, SSIM)を計算した。さらに単純差動回転モデルとの比較も行った。
      これらの値の時間経過に対する変化を観察し、より不確定性の高い将来の予測に対しても動画予測モデルが有効であるかを検証した。
      
      \subsubsection{平均輝度の再現}
        \paragraph{平均輝度の絶対誤差の計算}
          テストセット全体における、ある時間ステップtの平均輝度の絶対誤差を以下のように計算した。
          \begin{align}
            \bar{E}_{t} & = \frac{1}{N} \sum_{i=1}^{50} | \bar{I}_{\text{Prediction}_{i,t}} - \bar{I}_{\text{Actual}_{i,t}} |
          \end{align}
          ここで、iはテストセットのインデックスを表す。また、\( \bar{I}_{\text{Prediction}_{i,t}} \)は、テストセットi、時間ステップtにおける、モデルから生成された画像から計算された平均輝度を表し、\( \bar{I}_{\text{Actual}_{i,t}} \)は、実際の画像から計算された平均輝度を表す。
          平均輝度は全球(画像中の太陽の球面)に対してのみ行い、画像中の背景や外縁部からはみ出すコロナなどはその計算に含まれない。
          背景から全球に対して切り出される部分は、図\ref{fig:exp1_fulldisk_crop}に示されている。
          この全球の定義および計算は、取得したFITSファイルのヘッダーに記載される太陽の中心および半径に基づいている。
          \begin{figure}[htbp]
            \centering
            \includegraphics[width=0.6\textwidth]{figures/exp1/crop_map.png}
            \caption{生成した画像から全球部分のみ切り出した画像の例。この部分にのみ平均輝度を計算する。}
            \label{fig:exp1_fulldisk_crop}
          \end{figure}

          モデルの出力の全球での平均輝度と、実際の観測画像との誤差の推移を図\ref{fig:exp1_mean_intensity_line}に示す。
          これは、50のテストセットに対して、各テストセットに含まれる各画像の全球での平均輝度を計算し、その時間ステップごとの平均値を取ったものである。
          輝度の推移のみから特定の傾向を見出すことは難しいが、全体として平均絶対誤差は4\%以下に収まっている。
          \begin{figure}[htbp]
            \centering
            \includegraphics[width=\textwidth]{figures/exp1/error.png}
            \caption{MAUによるテストセットの予測画像と実際の観測画像の平均絶対誤差の時間推移。横軸が時間ステップ、縦軸が平均絶対誤差を表す。時間ステップを経る毎に誤差は単調に上昇していくが、その誤差は最終タイムステップでも5\%以下にとどまっている。}
            \label{fig:exp1_mean_intensity_line}
          \end{figure}
          
          さらに、入力シークエンスの最後から48時間後の画像の全球での平均輝度と、実際の観測画像との差異を観察する。その散布図\ref{fig:exp1_mean_intensity_scatter}に示す。
          このタイムステップは、出力の最後のタイムステップであり、最も不確定性の高い予測である。
          相関係数は0.97であり、非常に良好な値である。実際の観測画像の平均輝度の高低に関わらず、高い精度で平均輝度を再現できていることがわかる。
          \begin{figure}[htbp]
            \centering
            \includegraphics[width=0.6\textwidth]{figures/exp1/intensity_scatter_gt_pd.png}
            \caption{テストセットの最終ステップにおける全球平均輝度の予測対実測の散布図。縦軸がMAUによる予測から計算された平均輝度強度、横軸が実際の観測画像から計算された平均輝度強度を表す。計算された相関係数は0.97である。}
            \label{fig:exp1_mean_intensity_scatter}
          \end{figure}

        \paragraph{単純差動回転モデルとの比較}
          モデルの予測性能をさらに詳細に評価するために、シンプルな差動回転モデルとの比較を行った。
          目視や、平均輝度から、モデルの出力は実際の観測画像と概ね合致しており、特に自転による構造的変化などの主要な時間的特徴を再現できていることがわかった。
          ここでは、単純差動回転シミュレーションモデルによる出力と、我々のモデルの出力の再現精度の比較を行う。
          これにより、モデルが単に自転を予測しているのではなく、より複雑な時間的変化を予測できているかを検証した。

          太陽は自転するが、その実体は流体であるため、緯度によって自転速度が異なる。極付近の自転周期は約35日であるが、赤道付近では約25日である。この現象を差動回転と呼ぶ。
          差動回転をシミュレーションする研究は盛んに行われているが、ここでは回転速度を緯度依存としてモデル化するHoward et al. (1990) \cite{howard1990solar}の差動回転モデルを用いた。
          このモデルは、Heliographic緯度\(\theta\)に対する回転速度\(\omega(\theta)\)を以下のように定義する:
          
          \begin{align}
            \omega(\theta) &= A + B \sin^{2}(\theta) + C \sin^{4}(\theta) \\
            \text{where} \quad A &= 2.894 \, \mu\text{rad/s}, \\
            B &= -0.428 \, \mu\text{rad/s}, \\
            C &= -0.370 \, \mu\text{rad/s}
          \end{align}
          このモデルはSunpyに実装されており、\textit{physics.differential\_rotation}というモジュールとして提供されている。
          このモジュールによる画像の生成は、全球の各ピクセルに対して、そのピクセルの緯度に対する回転速度を計算し、その速度で西に向かって各ピクセルを移動させることで行われる。
          この単純差動回転モデルによるシミュレーションの例を図\ref{fig:exp1_sdr_example}に示す。
          比較は、単純差動回転モデルによるシミュレーションと、実際の観測画像との平均輝度の絶対誤差を計算し、それを前述の動画予測によるものと比較することで行った。
          この誤差の推移を図\ref{fig:exp1_sdr_line}に示す。単純差動回転モデルは時間経過とともに誤差が単調に増加するが、動画予測モデルは最終タイムステップにおいても誤差が4\%以下に収まっている。
          t=1 \~ 4の間は、単純差動回転モデルの方が誤差が小さいが、それ以降は逆転し、動画予測モデルの方が誤差が小さくなっている。
          このことから、動画予測モデルは、単純差動回転モデルよりも、より時間経過に対して堅牢であることがわかる。
          \begin{figure}[htbp]
            \centering
            \includegraphics[width=1.1\textwidth]{figures/exp1/sdr.png}
            \caption{Howard (1990)による差動回転モデルによるシミュレーションの例。入力シークエンスの最終入力(t=11)(左)をもとに、各ピクセルに式(4.2)を適用し移動させることで画像を生成する。全球面以外の背景には計算されない。}
            \label{fig:exp1_sdr_example}
          \end{figure}

          \begin{figure}[htbp]
            \centering
            \includegraphics[width=\textwidth]{figures/exp1/error_dr.png}
            \caption{MAUによるテストセットの予測画像と実際の観測画像の平均絶対誤差(オレンジ)と、単純差動回転モデルと実際の観測画像の平均絶対誤差(緑)。}
            \label{fig:exp1_sdr_line}
          \end{figure}
          
          さらに、出力シークエンスの最後のタイムステップにおいて、単純差動回転モデルによるシミュレーションと、実際の観測画像との差異を観察し、動画予測モデルによる出力と比較した。
          その散布図\ref{fig:exp1_sdr_scatter}に示す。
          単純差動回転モデルによるシミュレーションの平均輝度と、実際の観測画像の平均輝度は、相関係数では0.85である。
          データ点が全体的に左上に偏っていることから、単純差動回転モデルは、実際の観測画像よりも平均輝度を高く予測していることがわかる。
          一方で、前述のように、動画予測モデルによる出力の平均輝度と、実際の観測画像の平均輝度は、相関係数では0.97である。
          こことから、やはり最終タイムステップでの予測においても、動画予測モデルは単純差動回転モデルよりも高い精度で平均輝度を再現できていることがわかる。
          \begin{figure}[htbp]
            \begin{subfigure}[b]{0.55\textwidth}
              \centering
              \includegraphics[width=\textwidth]{figures/exp1/intensity_scatter_gt_pd.png}
              \caption{MAUによる、テストセットの最終ステップにおける全球平均輝度の予測対実測の散布図。計算された相関係数は0.97である。}
            \end{subfigure}
            \begin{subfigure}[b]{0.55\textwidth}
              \centering
              \includegraphics[width=\textwidth]{figures/exp1/intensity_scatter_gt_dr.png}
              \caption{単純差動回転モデルによる、テストセットの最終ステップにおける全球平均輝度の予測対実測の散布図。計算された相関係数は0.85である。}
            \end{subfigure}
            \label{fig:exp1_sdr_scatter}
            \caption{予測対実測の散布図。縦軸が予測から計算された平均輝度強度、横軸が実際の観測画像から計算された平均輝度強度を表す。}
          \end{figure}

      \subsubsection{画像類似度}
        画像内での構造的再現度とその時間的変化を評価するために、モデルの出力と対応する時間ステップの実際の観測画像の間のSSIMを計算した。
        SSIMは、画像の品質評価を目的として、Wang et al. (2004)\cite{wang2004image}で提案された。
        SSIMは特に構造情報が重要とされる医療画像や衛星画像のような分野で広く使用されている。従来の平均二乗誤差(MSE)やピーク信号対雑音比(PSNR)と比較して、SSIMは人間の視覚システムにより近い知覚品質を提供する。
        従来の手法とは異なり、SSIMは画像の輝度、コントラスト、構造の三つの比較を基にしている。
        SSIMの定義は以下の通りである:
        \begin{equation}
          SSIM(x, y) = \frac{(2\mu_x \mu_y + C_1)(2\sigma_{xy} + C_2)}{(\mu_x^2 + \mu_y^2 + C_1)(\sigma_x^2 + \sigma_y^2 + C_2)},     
        \end{equation}
        ここで、$x$と$y$は比較される二つの画像、$\mu_x$、$\mu_y$はそれぞれの画像の平均輝度、$\sigma_x^2$、$\sigma_y^2$はそれぞれの分散、$\sigma_{xy}$は共分散である。$C_1$と$C_2$は安定性のための小さな定数である。
        
        テストセット全体における、ある時間ステップtのSSIMの平均を以下のように計算した。
        \begin{align}
          \bar{SSIM}_{t} & = \frac{1}{N} \sum_{i=1}^{50} \text{SSIM}_{i,t}
        \end{align}

        画像類似度も、全球での平均輝度と同様に、全球に対してのみ行い、画像中の背景や外縁部からはみ出すコロナなどはその計算に含まれない。
        また、平均輝度の場合と同様に、単純差動回転モデルとの比較も同時に行った。
        このように計算されたSSIMの時間推移を図\ref{fig:exp1_ssim_line}に示す。
        MAUのSSIM、単純差動回転モデルによるSSIMは、共に時間経過とともに単調に減少していくが、MAUによるSSIMの方が、最終タイムステップにおいても0.94を超えているのに対し、単純差動回転モデルによるSSIMは0.92程度である。
        また、t=12\~16の初期段階では、単純差動回転モデルの方がSSIMが高いが、それ以降は逆転し、MAUによるSSIMの方が高い。
        この傾向は平均輝度の場合と概ね同様であり、やはりMAUによる予測は画像類似度においてもその低下が緩やかである。 
        
        \begin{figure}[htbp]
          \centering
          \includegraphics[width=\textwidth]{figures/exp1/average_ssim.png}
          \caption{テストセットでのSSIMの時間推移。SSIMは0から1の値を取り、二つの画像が類似するほど1に近づく。横軸が時間ステップ、縦軸がSSIMを表す。}
          \label{fig:exp1_ssim_line}
        \end{figure}
      
% *************************************************************************************************************

    \subsection{経度依存性の評価}
        さらに、予測性能が経度ごとにばらつきがあるかを確認するために、経度ごと予測の再現度を評価した。
        具体的には、Heliographic Stonyhurst座標系における経度-90°から90°までの半球を、36°ごとに5つのセクターに分割した。
        分割の概念図を図\ref{fig:exp1_division_concept}に示す。
        評価指標には、全球の場合と同様に、平均輝度の平均絶対誤差と、SSIMによる画像類似度を用いた。また、それぞれの評価において、単純差動回転モデルとの比較も行った。
        
        \begin{figure}[htbp]
          \centering
          \includegraphics[width=0.65\textwidth]{figures/exp1/devision_caption.jpg}
          \caption{分割の様子を示した図。Heliographic Stonyhurst経度-90°から90°までの半球を、36°ごとに5つのセクターに分割した。}
          \label{fig:exp1_division_concept}
        \end{figure}

        \subsubsection{平均輝度の再現}
          ここでは、全てのテストセットで各セクターごとの平均輝度を計算し、対応する時間ステップの実際の観測画像との間の絶対誤差を計算した。
          ここで、ある時間ステップt、ある経度セクターlにおける平均輝度の絶対誤差\( \bar{E}_{l,t} \)は以下のように定義される:
          \begin{align}
            \bar{E}_{l, t} & = \frac{1}{N} \sum_{i=1}^{50} | \bar{I}_{\text{Prediction}_ {i, l, t}} - \bar{I}_{\text{Actual}_{i, l, t}} | \\
          \end{align}
          ここで、iはテストセットのインデックスを表す。また、\( \bar{I}_{\text{Prediction}_{i, l, t}} \)は、テストセットi、時間ステップt、経度セクターlにおける予測された平均輝度を表し、\( \bar{I}_{\text{Actual}_{i, l, t}} \)は、実際の平均輝度を表す。  
            
          このように計算された誤差率の時間推移を図\ref{fig:lng_error}に示す。
          単純差動回転モデルとの比較を行っている。
          
          \paragraph{経度-90度から-54度}
          -90度から-54度のセクターは、東の外縁部(画像に向かって左側)の領域である。
          太陽は東から西に向かって回転する。そのため、時間が経過するにつれて、東側の外縁部では、太陽の新しい表面が全球面に観測されるようになる。
          またこのセクターでは、観測者である望遠鏡に対して角度がきつく、実際の太陽表面の面積に対して観測できる面積が小さい。
          しかし、時間が経過し表面が太陽の中心経度に向かうにつれて、観測される面積は大きくなることから、この領域を正確に予測するには、少ない情報から詳細な予測を行う必要がある。
          このようなタスクはしばしば困難であり、シンプルなモデルで正確に予測することは難しいと考えられる。
          この領域に対する誤差の推移を図\ref{fig:lng_error_1}から検証すると、MAUによる予測は、最終タイムステップで20\%程度と、全球平均の誤差率と比べると高くなっている。
          一方で、単純差動回転モデルによる予測は、最終タイムステップで45\%程度であり、また全タイムステップにおいて、MAUによる予測よりも誤差率が高くなっている。
          このことから、不確実性が高い東側外縁部の領域に対しても、MAUは有効な予測能力を持っていることがわかる。この点については、後述する東側外縁部に対する視覚的評価でも確認する。

          \paragraph{経度-54度から-18度}
          -54度から-18度のセクターは、東側の中心部の領域である。
          この領域は、東側外縁部と比べると、観測される面積が大きいため、より予測は容易になるものの、時間経過によって東側外縁部から移動してくる表面を予測しなければならないため、一定の難しさがある。
          ここでの誤差の推移を図\ref{fig:lng_error_2}から検証すると、MAUによる予測は、最終タイムステップで15\%程度と、全球平均の誤差率と比べるとやはり高くなっている。
          一方で、単純差動回転モデルによる予測は、最終タイムステップで30\%程度であり、また全タイムステップにおいて、MAUによる予測よりも誤差率が高くなっている。
          このことから、東側外縁部と比べると予測が容易であるこの領域でも、予測性能の時間推移は全体的な傾向と同様であることがわかる。

          \paragraph{経度-18度から18度}
          -18度から18度のセクターは、太陽の中心部の領域である。
          この領域は、太陽の中心部であり、観測される面積が最も大きいため、予測は比較的容易であると考えられる。
          ここでの誤差の推移を図\ref{fig:lng_error_3}から検証すると、MAUによる予測も単純差動回転モデルによる予測も、全タイムステップで10\%以下であり、予測性能は高いことがわかる。
          


          \begin{figure}[htbp]
            \begin{subfigure}{0.5\textwidth}
              \centering
              \includegraphics[width=\textwidth]{figures/exp1/lng_error_1.png}
              \caption{-90度から-54度}
              \label{fig:lng_error_1}
            \end{subfigure}
            \begin{subfigure}{0.5\textwidth}
              \centering
              \includegraphics[width=\textwidth]{figures/exp1/lng_error_2.png}
              \caption{-54度から-18度}
            \end{subfigure} \par
            \begin{subfigure}{0.5\textwidth}
              \centering
              \includegraphics[width=\textwidth]{figures/exp1/lng_error_3.png}
              \caption{-18度から18度}
            \end{subfigure}
            \begin{subfigure}{0.5\textwidth}
              \centering
              \includegraphics[width=\textwidth]{figures/exp1/lng_error_4.png}
              \caption{18度から54度}
            \end{subfigure} \par
            \begin{subfigure}{0.5\textwidth}
              \centering
              \includegraphics[width=\textwidth]{figures/exp1/lng_error_5.png}
              \caption{54度から90度}
            \end{subfigure}
            \caption{分割された各セクターにおける平均輝度の絶対誤差の時間推移。横軸が時間ステップ、縦軸が平均絶対誤差を表す。各グラフで縦軸の範囲が異なる。緑線がMAUによる予測から計算された絶対誤差、オレンジ線が単純差動回転モデルによるシミュレーションから計算された絶対誤差を表す。}
            \label{fig:lng_error}
          \end{figure}
          
        \subsubsection{画像類似度}
          全球での場合と同様に、経度ごとにも画像類似度を計算した。その時間推移を図\ref{fig:lng_ssim}に示す。同時に単純差動回転モデルの経度ごとの画像類似度も計算した。
          \begin{figure}[htbp]
            \begin{subfigure}{0.55\textwidth}
              \centering
              \includegraphics[width=\textwidth]{figures/exp1/lng_ssim_1.png}
              \caption{-90度から-54度}
            \end{subfigure}
            \begin{subfigure}{0.5\textwidth}
              \centering
              \includegraphics[width=\textwidth]{figures/exp1/lng_ssim_2.png}
              \caption{-54度から-18度}
            \end{subfigure} \par
            \begin{subfigure}{0.5\textwidth}
              \centering
              \includegraphics[width=\textwidth]{figures/exp1/lng_ssim_3.png}
              \caption{-18度から18度}
            \end{subfigure}
            \begin{subfigure}{0.5\textwidth}
              \centering
              \includegraphics[width=\textwidth]{figures/exp1/lng_ssim_4.png}
              \caption{18度から54度}
            \end{subfigure} \par
            \begin{subfigure}{0.5\textwidth}
              \centering
              \includegraphics[width=\textwidth]{figures/exp1/lng_ssim_5.png}
              \caption{54度から90度}
            \end{subfigure}
            \label{fig:lng_ssim}
            \caption{分割された各セクターにおけるSSIMの時間推移。横軸が時間ステップ、縦軸がSSIMを表す。各グラフで縦軸の範囲が異なる。緑線がMAUによる予測から計算されたSSIM、オレンジ線が単純差動回転モデルによるシミュレーションから計算されたSSIMを表す。}
          \end{figure}

% *************************************************************************************************************
    
    \subsection{東側外縁部に対する評価}
      ここまでで、作成した動画予測モデルは、全球での平均輝度や、経度ごとの平均輝度といった定量的な評価において、実際の観測画像を正確に再現できていることを確認した。
      既存のシンプルなシミュレーションモデルとの比較でも、平均輝度の評価においては、動画予測モデルの優位性を確認できた。

      動画予測モデルのシミュレーションモデルに対するさらなる独自の特徴として、望遠鏡の視野に入っていない太陽の球面を生成することができる点が挙げられる。
      Sunpyによって提供される差動回転シミュレーションモデル\textit{physics.defferential\_rotation}は、入力された画像の全球面の各ピクセルに対して差動回転を適用することで画像を生成する。
      そのため、入力時点で望遠鏡の視野に入っていない太陽の球面を生成することができないので、より長い時間スパンでの予測を行うと、東の外縁部から徐々に予測できない領域が広がっていく。
      これに対して、動画予測モデルは、入力画像の全球面に対して特定の数理モデルを適用するのではなく、過去のデータや全体的な文脈を元に、視野に入っていなかった領域を含む未来の状態を生成する。
      
      ここでは、動画予測モデルが、そのような「入力画像の時点で全球面に見えていない領域」に対して予測能力を持つか検証を行うため、 生成された画像の東側外縁部から出現する活動領域に対する評価を行う。

      \subsubsection{視覚的評価}
        ここでは、動画予測モデルが、東側外縁部から出現する活動領域に対して、どのような予測を行っているかを視覚的に評価する。
        いくつかの例を図\ref{fig:exp1_limb_example_1}および\ref{fig:exp1_limb_example_2}に示す。ここで示す画像は、左列が入力シークエンスの最終データ、中央列がその24時間後の予測画像、右列がその48時間後の予測画像である。

        \begin{figure}[htbp]
          \centering
          \includegraphics[width=\textwidth]{figures/exp1/limb_sample_3_caption.jpg}
          \caption{東側外縁部から出現する活動領域に対する予測の例。左列が入力シークエンスの最終データ、中央列がその24時間後の予測画像、右列がその48時間後の予測画像である。}
          \label{fig:exp1_limb_example_1}
        \end{figure}
        \begin{figure}[htbp]
          \centering
          \includegraphics[width=\textwidth]{figures/exp1/limb_sample_12_caption.jpg}
          \caption{東側外縁部から出現する活動領域に対する予測の例。左列が入力シークエンスの最終データ、中央列がその24時間後の予測画像、右列がその48時間後の予測画像である。}
          \label{fig:exp1_limb_example_2}
        \end{figure}

      \subsubsection{予測対実測散布図による定量的評価}
        さらに、東側外縁部に対する評価を行うために、予測対実測の散布図を作成した。その結果を図\ref{fig:exp1_limb_scatter}に示す。
        左は、MAUによる予測画像の東側外縁部の平均輝度と、実際の観測画像の東側外縁部の平均輝度の散布図である。
        実際の観測画像の東側外縁部の平均輝度と、その48時間後の予測画像の東側外縁部の平均輝度は、相関係数0.98で、強い相関があることがわかる。
        また、最終タイムステップにおける実際の観測画像の東側外縁部の平均輝度強度が、その48時間の値とどのように一致しているかを示す散布図も作成した。その結果を右に示す。
        この散布図から、東側外縁部の平均輝度と、その48時間後の東側外縁部の平均輝度は、相関係数0.26と相関が弱く、時間経過により容易に変化することがわかる。
        これは、東側外縁部の平均輝度の変化は、単純なロジックでは予測できないことを示している。
        
        これらの結果から、動画予測モデルは、一定の複雑さを持つ東側外縁部の平均輝度に対しても、高い予測能力を持つことがわかった。
        \begin{figure}[htbp]
          \begin{subfigure}[b]{0.53\textwidth}
            \centering
            \includegraphics[width=\textwidth]{figures/exp1/limb_scatter_gt_pd.png}
            \caption{すべてのテストセットの、最終タイムステップでの東側外縁部の平均輝度の予測対実測の散布図。横軸が実際の観測画像から計算された平均輝度強度、縦軸がMAUによる予測から計算された平均輝度強度を表す。計算された相関係数は0.98である。}
          \end{subfigure}
          \begin{subfigure}[b]{0.55\textwidth}
            \centering
            \includegraphics[width=\textwidth]{figures/exp1/limb_scatter_gt_sp.png}
            \caption{すべてのテストセットでの、最終タイムステップでの東側外縁部の平均輝度の実測値と、その48時間前の実測値の散布図。横軸が実際の観測画像から計算された平均輝度強度、縦軸がMAUによる予測から計算された平均輝度強度を表す。計算された相関係数は0.26である。}
          \end{subfigure}
          \caption{}
          \label{fig:exp1_limb_scatter}
        \end{figure}
    
    \subsection{まとめ}
      
      本実験での結果を表\ref{tab:exp1_result}にまとめる。
      \begin{table}[htbp]
        \centering
        \caption{本実験での各評価の結果。MAUは、本研究で使用した動画予測モデルによる予測に対する評価、Howard (1990)は、単純差動回転モデルによるシミュレーションに対する評価を表す。}
        \begin{tabular}{lcccccc}
        \hline
        評価指標 & 全球 & \multicolumn{5}{c}{経度ごと} \\
        \cline{3-7}
         &  & -90 to -54 & -54 to -18 & -18 to 18 & 18 to 54 & 54 to 90 \\
        \hline\hline
        平均輝度絶対誤差↓ & & & & & & \\
        \quad MAU - 1波長 & 0.05 & 0.04 & 0.03 & 0.05 & 0.06 & 0.04 \\
        \quad Howard (1990) & 0.06 & 0.05 & 0.04 & 0.06 & 0.07 & 0.05 \\
        \hline
        SSIM↑ & & & & & & \\
        \quad MAU - 1波長  & 0.9 & 0.88 & 0.89 & 0.87 & 0.85 & 0.86 \\
        \quad Howard (1990) & 0.85 & 0.87 & 0.86 & 0.84 & 0.83 & 0.85 \\
        \hline
        \end{tabular}
        \label{tab:exp1_result}
      \end{table}
       

  \section{考察}


\chapter{Motion-Aware Unitを用いた3波長を入力とした紫外線像の全球時系列予測}
  \section{実験概要}
    ここでは、171Å、193Åフィルターで得られたデータを追加で利用し、3波長の入力データから211Åの波長データに対する予測を行った。
    これらの波長は、太陽のコロナ領域における異なる温度帯を観測するためのものであり、予測モデルに多様な物理的情報を提供することが期待される。
    171Åの波長は、太陽のコロナにおける温度が約60万Kの領域を捉えるのに特化しており、193Åの波長は約100万Kの領域を捉える。
    これらの波長から得られる情報を組み合わせることにより、単一の波長では捉えられない層間の相互作用を捉え、より高い精度での予測を可能にすることを期待する。
    
    モデルには先の実験と同じく、MAUを用いる。入力は3波長、すなわち画像的には3チャンネルである。
    目的となる出力は211Åの波長のみであるが、MAUは3チャンネルを出力する。
    これは、「出力シークエンスのタイムステップ1以降では、直前のモデルの出力を入力データとして扱う」という動画予測モデルの一般的な性質によるものである。
    このような性質から、211Åの波長のみを出力として扱うために、出力された3チャンネルのうち、211Åの波長のみを抽出するという処理を行った。
  
  \section{実験設定}
    各ハイパーパラメータの設定を表\ref{tab:exp1_hyperparameters}に示す。チャンネル数のみ前回の実験から変更されている。
    \begin{table}[h]
      \centering
      \begin{tabular}{lc}
      \hline
      ハイパーパラメータ & 値 \\
      \hline\hline
      バッチサイズ & 4 \\
      \hline
      エポック数 & 100 \\
      \hline
      学習率 & 0.0005 \\
      \hline
      損失関数 & MSE \\
      \hline
      チャンネル & 3 \\
      \hline
      カーネルサイズ & (5, 5) \\
      \hline
      MAU Cell数 & 16 \\
      \hline
      \end{tabular}
      \caption{本実験でのハイパーパラメータ設定。基本的には前実験と同様であるが、チャンネル数が1から3に変更されている。}
      \label{tab:exp1_hyperparameters}
    \end{table}

    データに関しても、データ数の増減による影響がないように、前回の実験と同じ期間のデータを用いた。欠損期間なども同様である。

  \section{学習の推移}
  
  学習は図\ref{fig:exp2_learn_progress}のように推移した。学習の初期段階では、学習データに対する損失関数の値が急激に減少しているが、学習が進むにつれて収束に向かって緩やかに減少していることがわかる。
  また、学習損失、検証損失ともに、安定的に減少していることがわかる。
  学習にはNVIDIA RTX A6000を用い、完了までに約31時間を要した。
  \begin{figure}[htpb]
    \centering
    \includegraphics[width=\textwidth]{figures/mau.png}
    \caption{本実験での、学習データ、検証データでの損失関数の推移。どちらも安定的に減少している。}
    \label{fig:exp2_learn_progress}
  \end{figure}

  \section{実験結果}
    図\ref{fig:exp2_out}に、この実験での出力例を示す。
    \begin{figure}[h]
      \centering
      \includegraphics[width=\textwidth]{figures/mau.png}
      \caption{}
      \label{fig:exp2_out}
    \end{figure}
    モデルの出力は、視覚的には実際の観測画像と概ね合致している。
    この実験における評価では、前回の実験と同様の評価を行った。
  
    \subsection{全球での評価}
      はじめに全球での評価を行った。
      前回実験と同様に、まず輝度強度の平均値と実際の平均値との誤差、SSIMを計算した。さらに単純差動回転モデルとの比較も行った。
      また、これらの値の時間経過に対する変化を観察し、より不確定性の高い将来の予測に対しても動画予測モデルが有効であるかを検証した。

      \subsubsection{平均輝度とその誤差}
        モデルの出力の全球での平均輝度と、実際の観測画像との誤差の推移を図\ref{fig:exp2_mean_intensity_line}に示す。
        これは、50のテストセットに対して、各テストセットに含まれる各画像の全球での平均輝度を計算し、その時間ステップごとの平均値を取ったものである。

        \begin{figure}[h]
          \centering
          \includegraphics[width=\textwidth]{figures/mau.png}
          \caption{}
          \label{fig:exp2_mean_intensity_line}
        \end{figure}
        
        さらに、入力シークエンスの最後から48時間後の画像の全球での平均輝度と、実際の観測画像との差異を観察する。その散布図\ref{fig:exp2_mean_intensity_scatter}に示す。
        このタイムステップは、出力の最後のタイムステップであり、最も不確定性の高い予測である。

        \begin{figure}[h]
          \centering
          \includegraphics[width=\textwidth]{figures/mau.png}
          \caption{}
          \label{fig:exp2_mean_intensity_scatter}
        \end{figure}

      \subsubsection{画像類似度}
        前回実験と同様に、画像内での構造的再現度とその時間的変化を評価するために、モデルの出力と対応する時間ステップの実際の観測画像の間のSSIMを計算した。
        SSIMの推移を図\ref{fig:exp2_ssim_line}に示す。画像類似度は、全球での平均輝度と同様に、全球に対してのみ行い、画像中の背景や外縁部からはみ出すコロナなどはその計算に含まれない。
        \begin{figure}[h]
          \centering
          \includegraphics[width=\textwidth]{figures/mau.png}
          \caption{}
          \label{fig:exp2_ssim_line}
        \end{figure}

      \subsubsection{単純差動回転モデルとの比較}
        前回実験と同様に、モデルの予測性能をさらに詳細に評価するために、シンプルな差動回転モデルとの比較を行った。
        比較は、単純差動回転モデルによるシミュレーションと、実際の観測画像との間の平均輝度の絶対誤差を計算し、それを前述の動画予測によるものと比較することで行った。
        この誤差の推移を図\ref{fig:exp2_sdr_line}に示す。

        \begin{figure}[h]
          \centering
          \includegraphics[width=\textwidth]{figures/mau.png}
          \caption{}
          \label{fig:exp2_sdr_line}
        \end{figure}
        
        さらに、出力シークエンスの最後のタイムステップにおいて、単純差動回転モデルによるシミュレーションと、実際の観測画像との差異を観察し、動画予測モデルによる出力と比較した。
        その散布図\ref{fig:exp2_sdr_scatter}に示す。
        
        \begin{figure}[h]
          \centering
          \includegraphics[width=\textwidth]{figures/mau.png}
          \caption{}
          \label{fig:exp2_sdr_scatter}
        \end{figure}
        
      \subsection{経度依存性の評価}
        前回実験と同じく、予測性能が経度ごとにばらつきがあるかを確認するために、経度ごと予測の再現度を評価した。分割の方法は前回実験と同様である。
        評価指標には、平均輝度の誤差と、その単純差動回転モデルとの比較を用いた。

        \subsubsection{平均輝度とその誤差}
          ここでは、全てのテストセットで各セクターごとの平均輝度を計算し、対応する時間ステップの実際の観測画像との間の絶対誤差を計算した。
          誤差率の時間推移を図\ref{fig:exp2_mean_intensity_longitude_line}に示す。
          \begin{figure}[h]
            \centering
            \includegraphics[width=\textwidth]{figures/mau.png}
            \caption{}
            \label{fig:exp2_mean_intensity_longitude_line}
          \end{figure}
          
          さらに、全球での評価と同様に、出力シークエンスの最後のタイムステップにおいて、動画予測モデルの出力から計算される経度ごとの平均輝度と、実際の観測画像での経度ごとの平均輝度をプロットした散布図を図\ref{fig:exp1_mean_intensity_longitude_scatter}に示す。
          
          \begin{figure}[h]
            \centering
            \includegraphics[width=\textwidth]{figures/mau.png}
            \caption{}
            \label{fig:exp2_mean_intensity_longitude_scatter}
          \end{figure}

        \subsubsection{単純差動回転モデルとの比較}
          全球での場合と同様に、経度ごとの比較でも単純差動回転モデルとの比較を行った。その時間推移を図\ref{fig:exp1_sdr_longitude_line}に示す。
          \begin{figure}[htbp]
            \centering
            \includegraphics[width=\textwidth]{figures/mau.png}
            \caption{}
            \label{fig:exp1_sdr_longitude_line}
          \end{figure}
          
          また、出力シークエンスの最後のタイムステップにおける、経度ごとの単純差動回転モデルによるシミュレーションと、実際の観測画像との差異を観察し、動画予測モデルによる出力と比較した。
          その散布図を図\ref{fig:exp1_sdr_longitude_scatter}に示す。
          \begin{figure}[htbp]
            \centering
            \includegraphics[width=\textwidth]{figures/mau.png}
            \caption{}
            \label{fig:exp1_sdr_longitude_scatter}
          \end{figure}

    \subsection{東側リムから出現する活動領域に対する視覚的評価}
  \section{考察}

\chapter{議論}

\section{全体的な考察}
  未知の太陽全球画像を予測することは、既存の予測モデルの予測能力を拡張し、専門家にとっても有用な情報源を提供する可能性があるため、より早期の宇宙天気予報の実現において有用である。
  本研究では、深層学習を用いた動画予測手法を用いて、SDO / AIAの時系列画像から、48時間以内の全球紫外線画像を生成することを目的とした。

  はじめに、AIAの211\AA フィルターから得られた時系列全球データを入力とし、MAUを用いて48時間以内の4時間ごとの全球紫外線画像を生成するモデルを構築した。
  このモデルを用いた実験では、生成された画像に対して、全球、経度ごと、さらに東側外縁部における輝度強度の再現性を評価した。
  さらに、単純作動回転モデルとの比較を行い、既存のシミュレーションモデルに対する性能を評価した。
  この実験では、MAUは各評価指標のほとんど全てにおいて単純差動回転モデルを上回り、また時空間的なロバスト性を持つことを確認することができた。
  
  次に、AIAの211\AA, 193\AA, 171\AA フィルターから得られた時系列全球データを入力とし、MAUを用いて48時間以内の4時間ごとの全球紫外線画像を生成するモデルを構築し、同じく評価を行った。
  この実験では、ほとんどの評価指標において、実験1の結果を下回るか、同等の結果となった。
  これは、現在使用してるMAUのアーキテクチャやモデルの深さでは、3波長の入力に対して十分な学習を行うことができないことが原因と考えられる。
  
  以上の結果から、本研究で構築したMAUを用いた動画予測モデルは、既存の予測モデルの予測能力を拡張することができることが示された。
  特に、以下の点については、本研究で用いた動画予測モデルの注目すべき強みであると言える。

  \paragraph{空間的ロバスト性}
    経度ごと評価や、東側外縁部における評価において、MAUは低い誤差を維持し、単純差動回転モデルよりも優れた性能を示した。
    東側外縁部以外の領域では、その面の角度とそれによる歪みの有無にかかわらず、MAUはほとんど同じ精度で予測を行うことができる。
    さらに、常に新しい面が観測される東側外縁部においても、MAUは間接的な情報を利用して予測を行ことができると考えられ、高い学習能力を持つことが示唆される。
    
    この高い学習性能とロバスト性は、全球を偏りなくバランスよく再現できるということであり、動画予測モデルの重要な特徴である。
    太陽表面での現象は、その位置により、地球への影響の程度が異なるため、全球の情報を正確に捉えることは重要である。
    そのような課題に対して、全球の広範囲にわたって正確に予測できるこのモデルは、宇宙天気予報において有用であると考えられる。
  
  \paragraph{時間的ロバスト性と確率的予測}
    動画予測モデルは、ほとんどの評価指標において、時間経過に伴う性能の低下が、単純差動回転モデルよりも緩やかであった。
    これは、深層学習の確率的モデリングの特徴と、太陽という複雑な系の相性が良いことが要因に考えられる。
  
  \paragraph{高速な予測}
    本研究で用いた動画予測モデルであるMAUは、その学習の完了に10時間単位の計算時間と高性能なGPUを要求するが、学習済みモデルによるテストデータに対する予測は数秒で完了する。
    これは、スーパーコンピュータレベルの計算リソースを必要とする物理シミュレーションモデルと比較して非常に高速であり低コストである。
    この点は、迅速な予測が求められる宇宙天気予報において重要であり、動画予測モデルの高い有効性を示すものである。

\section{今後の展望と課題}
  本実験の結果は、宇宙天気予報における多くの新しいアプローチの可能性を示唆するものであると言える。
  下に示すようなさらなる動画予測モデルの改良や、実際の宇宙天気予報モデルへの直接的な応用により、本研究の成果をさらに発展させることができると考えられる。
  
  \subsection{異なるサンプリング間隔での予測}
    本研究では、4時間おきのデータを入力として48時間以内の予測を行った。
    これは、数日単位での全球紫外線画像の予測を目的としているためであるが、さらに高い時間分解能での短い時間スケールでの予測や、逆に長い時間スケールでの予測も有用であると考えられる。

    短い時間スケールでは、活動領域などに限定した予測を行うことなどが考えられる。
    例えば、フレアの発生を予測するために、活動領域に予測範囲を限定し、より高いサンプリング間隔での予測を行うことが考えられる。
    フレアの発生は非常に複雑な現象であるため、本研究の予測モデルで十分な精度で予測できるかは不明である。
    しかし、後述するモデル変更などのアプローチを行うことで、予測の精度を向上させる可能性がある。

    長い時間スケールでは、コロナホールの形状の変化の予測などが考えられる。
    コロナホールは全球で観測される中でも大規模な構造であり、その形状の変化は、活動領域などに比べてゆっくりとした時間スケールで起こる。
    ある特定のコロナホールに対して、自転周期程度の時間をサンプリング間隔として予測を行うことで、数ヶ月先までのコロナホールの形状を予測することができる可能性がある。
    このような予測は、宇宙天気予報において、コロナホールによる高速太陽風の到達を予測するために有用であると考えられる。

  \subsection{より表現力の高い動画予測モデルによる予測}
    本研究では、CNNと再帰的ニューラルネットワークを組み合わせた動画予測モデルであるMAUを用いて予測を行った。
    しかし、近年の動画予測モデルの研究では、より表現力や精度の高いモデルが提案されている。
    特に、\citex{dosovitskiy2020image}らにより発表された、Transformerをベースとした画像処理モデルであるVision Transformer (ViT) の登場以降、Transformerをアーキテクチャの中核に置いた動画予測モデルの研究が注目を集めている (\citex{li2022efficient}, \citex{tang2023swinlstm} )。
    Transformerは、その注意構造から、解釈可能性の高いモデルとしても注目されており、モデルの予測の理由を解析することができる。
    それにより、現象の予測だけでなく、その現象のダイナミクスの解明にも役立つ可能性がある。

  \subsection{異なる観測データでの予測}
    本研究では、SDO / AIAで観測される太陽の遷移層からコロナの領域を捉える全球紫外線像を入力として予測を行った。
    本研究では、少なくともAIA 211\AA フィルターのデータを入力とした場合、有効な予測結果を得ることができた。
    さらなる宇宙天気予報への応用と拡張として、他の波長で観測される紫外線画像や、磁場データなどを入力とした予測を行うことが考えられる。
    例えば、SDO / HMIで観測される磁場データは、フレア予測において最も重要なデータの一つであり、これを予測対象とすることは、より直接的な宇宙天気予報への応用となる。
    また、黒点の成長予測など、より難しい予測への挑戦も有用である。

    このように、動画予測モデルは用いた予測は、太陽における多くのイベント、現象に対する汎用性があり、本研究で示された可能性はまだその一部に過ぎないと考えられる。

  \subsection{実際の宇宙天気予測モデルへの応用}
    本研究では、実際の宇宙天気予報モデルの予測能力の拡張や、専門家による宇宙天気予報への情報源の提供を将来的な目標としつつ、その前段として、輝度強度の再現度を評価することで、動画予測モデルの有効性を検証した。
    その評価の結果、深層学習を用いた動画予測モデルは、目的とする全球紫外線像を精度よく再現した。

    実際の宇宙天気予報モデルへ入力データとし、その予測性能が拡張可能であるかどうかは、より詳細で実際的な評価が必要である。
    今後、そのような評価を行うことで、本研究の成果を実際の宇宙天気予報モデルへの応用に繋げることができると考えられる。
  

\chapter{結論}
  本研究では、より早期の正確な宇宙天気予報の実現に貢献するために、深層学習を用いた動画予測手法によって、未知の太陽全球画像を予測することを目的とした。
  SDO / AIAの211Åフィルターから得られた全球紫外線像を、Motion-Aware Unitによって予測するモデルを構築し、さまざまな条件下での評価を行った。
  実験の結果、本研究で構築した動画予測モデルは、目的とする全球紫外線像の輝度を精度よく再現した。
  また、高い時空間的ロバスト性を持ち、間接的な情報を利用して予測を行っていることも示唆され、高い学習能力を持つことが確認された。
  さらに、動画予測モデルは非常に高速に予測画像を生成することができた。

  これらの結果から、動画予測モデルは宇宙天気予報において有用な情報源を提供する可能性があることが示された。
  今後、他の観測データへの応用や、さらなる強力なモデルの構築、実際の宇宙天気予報モデルへの入力など、さらなるアプローチの探索を行うことで、本研究の成果をさらに発展させることができると考えられる。
\chapter*{謝辞}
  本研究は、筆者が新潟大学大学院自然科研究科に在学中に、飯田研究室において、飯田佑輔准教授のご指導のもの行われたものです。
  本研究のテーマは、宇宙天気の分野でも挑戦的なものであり、動画予測という手法もまだまだ新しい手法です。
  最初期の段階では、私の動画予測という手法を使いたいという動機から決定したテーマで、どのようなデータでどのような予測を行うか、どのような評価を行うかなど、多くの課題がありました。
  そのような中で、飯田准教授は、私の研究の方向性を見極め、的確なアドバイスを与えてくださいました。
  紫外線像での予測を行うこと、輝度強度の誤差率を評価すること、単純差動回転モデルとの比較を行うことなど、本研究の成果を大きく左右する重要なアドバイスを与えてくださいました。
  また、天文学会やJpGU、SGEPSS、Hinode-16 / IRIS-13 meetingなど、国際学会を含む多くの学会において、私の研究の発表の機会を与えてくださり、非常に多くのサポートをしてくださいました。
  学会での発表はプレッシャーも大きいものでしたが、他の学生や機関の研究者の方との交流は非常に有意義なものであり、モチベーションになりました。
  こうした飯田准教授のご指導のおかげで、本研究を進めることができました。
  この場を借りて、深く感謝申し上げます。
  
  国立研究開発法人情報通信研究機構 (NICT) の西塚直人様には、本研究の実施にあたり、多くの助言をいただきました。
  特に、実際に運用されている深層学習モデルのDeep Flare Netの開発者として、動画予測を用いた宇宙天気予報の可能性について、多くの示唆をいただきました。

  また、飯田研究室の先輩方、同期の方々には、研究の進め方や、学会発表の仕方など、多くのアドバイスをいただきました。
  特に、JpGUでの発表においては、初めての国際学会での英語発表であり、非常に緊張しましたが、ホテルで同室であった佐藤くんとは、同じ境遇であることを励まし合い、発表に臨むことができました。
  あの数日は、3年間の研究生活の中でも、最も濃厚な時間だったと思います。
  
  最後に、経済的にも精神的にも支えてくださった家族、親戚に心より感謝します。

  多くの方々のご協力のおかげで、本研究を進めることができました。
  この場を借りて、改めて深く感謝申し上げます。

\bibliographystyle{unsrtnat} % natbib互換のスタイル
\bibliography{references} % .bibファイルの名前

\end{document}
